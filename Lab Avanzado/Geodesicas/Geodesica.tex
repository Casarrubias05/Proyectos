\documentclass[11pt]{article}
\usepackage{amssymb, amscd, amsthm, amsfonts, mathtools, mathrsfs, tensor, braket, etoolbox}
\mathtoolsset{showonlyrefs,showmanualtags}
\usepackage{graphicx, wrapfig}
\usepackage{hyperref}
\hypersetup{
colorlinks=true,
urlcolor = cyan,
linkcolor=blue,
citecolor=brown}
%\hypersetup{hidelinks} %hides red outline around links
\usepackage[spanish,es-nodecimaldot, es-tabla,es-lcroman]{babel}
\usepackage{siunitx}
%\sisetup{separate-uncertainty}
\DeclareSIUnit{\atm}{\text{atm}}

\usepackage{upgreek}
\usepackage[minimal = true]{chemmacros}

\usepackage{cancel}
\usepackage{multicol}

\usepackage[lastexercise]{exercise}
\renewcommand{\AtBeginExercise}{\itshape}

\usepackage{titlesec}
 \numberwithin{equation}{section}

\usepackage{booktabs}

\setlength{\headheight}{14pt}

\usepackage{lastpage}

\usepackage[margin=2.5cm]{geometry}

\usepackage{esint}
\usepackage{booktabs}


\usepackage{enumerate}
\usepackage{nicefrac}


%\usepackage{mathpazo}

\usepackage{caption}
\usepackage{subcaption}

\usepackage{esint, csquotes}
\usepackage[
	style=apa
]{biblatex}
\addbibresource{Bibliografia.bib}
\usepackage{xurl}

\oddsidemargin 0pt
\evensidemargin 0pt
\marginparwidth 40pt
\marginparsep 10pt
\topmargin -20pt
\headsep 10pt
\textheight 8.7in
%\textwidth 6.65in

\renewcommand{\vec}{\mathbf} %Reescribe a los vectores en negritas y no con una flechita
\newcommand{\uniSym}[1]{\, \hat{\boldsymbol{#1}}}
\newcommand{\univ}[1]{\hat{\mathbf{e}}_{#1}} % Facilita la escritura de vectores unitarios
\newcommand{\basis}[1]{\mathbf{e}_{#1}}
\newcommand{\eqcomma}{\,  ,}
\newcommand{\R}{\mathbb R}
\renewcommand{\r}{\vec r}
\newcommand{\eqperiod}{\,  .}
\newcommand{\bfgreek}[1]{\bm{#1}}
\newcommand{\bgamma}{\boldsymbol{\gamma}}
\newcommand{\surface}{\mathcal S}
\newcommand{\patch}{\boldsymbol{\sigma}}
\newcommand{\tangent}{\boldsymbol{\mathfrak{t}}}
\newcommand{\normal}{\boldsymbol{\mathfrak{n}}}
\newcommand{\binormal}{\boldsymbol{\mathfrak{b}}}
\newcommand{\jacobian}[2]{\frac{\partial(#1)}{\partial(#2)}}
\newcommand{\EC}[1][1]{\frac{#1}{4\pi\epsilon_0}}


\newcommand{\crossp}[2]{\vec{#1} \times \vec{#2}}

\DeclarePairedDelimiter{\abs}{\lvert }{\rvert}
\DeclarePairedDelimiterX{\norm}[1]{\lVert}{\rVert}{\ifblank{#1}{\:\cdot\:}{#1}}
\DeclarePairedDelimiterXPP{\lnorm}[2]{}{\lVert}{\rVert}{_\ifblank{#2}{2}{#2}}{\ifblank{#1}{\: \cdot \:}{#1}}
\DeclarePairedDelimiterX{\innerp}[2]{\langle}{\rangle}{\ifblank{#1}{\:\cdot\:}{#1} \delimsize \vert \mathopen{} \ifblank{#2}{\:\cdot\:}{#2}}
\DeclarePairedDelimiterX{\scalarp}[2]{\langle}{\rangle}{#1 , #2}

\newcommand{\ChrisSym}[2]{\genfrac{\{}{\}}{0pt}{}{#1}{#2}}
\renewcommand{\ChrisSym}[2]{{\Gamma^{#1}}_{#2}}
\DeclareMathOperator{\Span}{span}

\newcommand{\Riemann}[2]{{R^{#1}}_{#2}}
\renewcommand{\Riemann}[2]{R\indices{^{#1}_{#2}}}

\newcommand{\partialD}[2]{\frac{\partial #1}{\partial #2}}
\newcommand{\derivative}[2]{\frac{d #1}{d #2}}

\newcommand{\Matrix}[1]{\begin{pmatrix}#1\end{pmatrix}} 

\renewcommand\qedsymbol{$\blacksquare$}%Hace un cuadrado negro el qed de una demostración
\newtheorem{theorem}{Teorema}[section] %Crea el entorno de teorema
\newtheorem{proposition}[theorem]{Proposición}
\newtheorem{corollary}[theorem]{Corolario}
\newtheorem{lemma}[theorem]{Lema}
\newtheorem{definition}[theorem]{Definición}
\newtheorem{remark}[theorem]{Observación}

\newcommand{\eto}[1]{e^{#1}}
\newcommand{\dgr}{^{\dagger}}
\renewcommand{\div}[1]{\nabla \cdot \vec{#1}}
\newcommand{\rot}[1]{\nabla \times \vec{#1}}
\newcommand{\iGamma}[2][\infty]{\int_{0}^{#1} t^{#2}\eto{-t} \, dt}

\DeclareMathOperator{\arccot}{arccot}
\DeclareMathOperator{\arcsenh}{arc\ senh}

\usepackage{fancyhdr}

\renewcommand{\footrulewidth}{0.4pt}
\pagestyle{fancy}

\fancyhead[L]{}
\fancyhead[R]{\thepage\ de \pageref{LastPage}}


\fancyfoot{}
\fancyfoot[C]{Luis Arturo Ureña Casarrubias}
%\fancyfoot[C]{\small Fernanda Arely Durán Ramirez}
%\fancyfoot[R]{\small José Emmanuel Chávez Zepeda}


\title{Laboratorio Avanzado\\
{\Large Geodésicas en Superficies}}
\author{Luis Arturo Ureña Casarrubias}
\date{\today}


\begin{document}

\maketitle

\tableofcontents

\newpage

\section{Introducción}
En el mundo tridimensional euclidiano (plano) en que habitamos, el concepto de \emph{geodésica} es fácil de explicar y aún más fácil de realizar: es la línea más corta que une dos puntos, y se puede ``hallar'' expéditamente: traza una línea recta que una a los dos puntos. Si la cuestión de hallar esta línea recta se le planteara a alguien con conocimientos elementales de física, contestaría:
\begin{quotation}
	Suponiendo que tú y una pelota se encuentren en un vacío (región del espacio ausente de materia y energía adicional), coloca a la pelota en el punto inicial $p$ y dale un empujón en dirección al punto final $q$. La trayectoria que recorrera es una línea recta.
\end{quotation}
Todo esto bajo la suposición de encontrar un vacío ideal como se ha descrito. Y más aún, todo lo discutido hasta ahora bajo el supuesto de que estamos tratando con un mundo euclidiano.

En los espacios o superficies curvos, que es en esto últimos en lo que nos enfocaremos en este trabajo, es de interés, tanto práctico como puramente matemático, generalizar la noción de \emph{línea recta} a estos objetos. En específico, la propiedad \emph{minimizadora} sobre la distancia que posee.

Existen otras caracterizaciones de las curvas geodésicas en una superficie bidimensional, la más sencilla de ellas es una curva cuya aceleración es siempre perpendicular a la superficie. La intuición detrás de esta definición es: imagínemonos que un conductor en su coche busca seguir una ``línea recta'' desde su perspectiva sobre la superficie, manteniendo una rapidez constante; para conseguir esto no debe girar ni a su derecha ni a su izquierda. Pero si su recorrido es curvo, por estar sobre una superficie curva, y no acelera en la dirección de su movimiento, la única dirección hacia la cual se puede doblar es ``\emph{hacia afuera}'' de la superficie. Entonces su aceleración, si es que la tiene, debe ser ortogonal a la superficie que recorre.

Esta y otras caracterizaciones que requieren conceptos adicionales, son equivalentes, por lo que nosotros trabajaremos con una geodésica como la curva que une dos puntos y tiene la menor longitud posible.

\section{Superficie de Trabajo}
Para hablar de geodésicas en una superficie, antes hay que hablar de curvas en una superficie, y antes de eso hay que definir qué son una curva y una superficie en el espacio $\R^3$ euclidiano.

\subsection{Curvas}

\parencite{tapp-2016} hace la siguiente definición:
\begin{definition}\label{def: curve}
	Sea $I \subset \R$ un intervalo. Una \emph{curva parametrizada} en $\R^n$ es una función suave (posiblemente a trozos) $\vec c : I \to \R^n$.
\end{definition}
Por convención, a una curva parametrizada se le llama simplemente una \emph{curva}, y su forma explícita es
\begin{equation}
	\vec c(\lambda) = \bigl(c^1(\lambda), \dotsc , c^n(\lambda) \bigr), \qquad t \in I.
\end{equation}
Aquí la notación $c^\mu$, consistente con análisis tensorial, se refiere a la $\mu$-ésima componente de $\vec c$ y no a la $\mu$-ésima potencia de $c$.

Notése que una curva en $\R^3$ se obtiene de la definición \ref{def: curve} simplemente con $n = 3$.

Que una curva sea suave quiere decir que su derivada $\dot{ \vec c}$ en $\lambda \in I$, definida como
\begin{equation}
	\dot{\vec c}(\lambda) \coloneqq  \lim_{h \to 0} \frac{\vec c(\lambda + h) - \vec c(\lambda)}{h},
\end{equation}
está bien definida en todo $I$, y que la derivada de $\dot{\vec c}$ también existe, así como todas las derivadas sucesivas. A la primera derivada $\dot{\vec c}$ se le llama \emph{velocidad} de $\vec c$, y a la segunda derivada $\ddot{\vec c}$ se le conoce como \emph{aceleración} de $\vec c$. Dada su forma explícita, si $\vec c$ es suave, entonces sus funciones componentes $c^\mu, \mu = \overline{1, n}$ también son suaves y
\begin{equation}
	\dot{\vec c}(\lambda) = \bigl(\dot c^1(\lambda), \dotsc , \dot c^n(\lambda) \bigr).
\end{equation}

Establecemos ahora la rapidez de $\vec c$: la rapidez de $\vec c$ en $t \in I$ es la función $s : I \to \R^+$ definida como 
\begin{equation}
	s(\lambda) \coloneqq \norm{\dot{\vec c}(\lambda)},
\end{equation}
donde $\norm{}$ es la norma estándar euclidiana $\lnorm{}{}$. 
De principal importancia son las curvas ``que no se detienen'' y las curvas de velocidad unitaria.
\begin{definition}
	Sea $\vec c$ una curva en $\R^n$ y $\lambda \in I$ arbitraria.
	\begin{enumerate}[(i)]
		\item $\vec c$ es \emph{regular} si satisface $s(\lambda) \neq 0$.
	
		\item $\vec c$ está \emph{normalizada} o \emph{parametrizada por longitud de arco} si $s(\lambda) = 1$.
	\end{enumerate}
\end{definition}
Estamos ahora en posición de hacer la definción central a este trabajo.
\begin{definition}\label{def: arcL}
	La longitud de arco de $\vec c$ entre $a, b \in I$ está definida como
	\begin{equation}
		\mathscr L(\vec c; a, b) \coloneqq \int_{a}^{b} \norm{\dot{\vec c}(\lambda)} \, d\lambda
		= \int_{a}^{b} s(\lambda) \, d\lambda.
	\end{equation}
\end{definition}
	Notemos que si fijamos $a \in I$ y definimos
	\begin{equation}
		L(\lambda) \coloneqq \mathscr L(\vec c; a, \lambda) = \int_{a}^{\lambda}s(\mu) \, d\mu,
	\end{equation}
por el primer teorema fundamental del cálculo obtenemos una función diferenciable, pues \\ $\dot L(\lambda)~=~s(\lambda)$. Otra observación es que si $\vec c$ es una curva con velocidad $s$ constante, entonces $\mathscr L(\vec c; a, b) = (b - a)s$. Concretamente, si $\vec c$ está normalizada, $\mathscr L(\vec c; a, b) = b - a$.

Por brevedad, cuando del contexto sea claro sobre qué intervalo se está calculando la longitud de arco de $\vec c$, denotaremos a esta simplemente por $\mathscr L(\vec c)$.

Una propiedad importante de la longitud de arco es que es independiente de cómo esté parametrizada la curva.
\begin{definition}
	Sea $\vec c : I \to \R^n$ regular. Una reparametrización de $\vec c$ es una función $\tilde{\vec c} : \tilde{I} \subset \R \to \R^n$ definida como $\tilde{\vec c} \coloneqq \vec c \circ \phi$, donde $\phi : \tilde{I} \to I$ es una biyección suave cuya derivada no se anula en ningún punto de $\tilde{I}$.
\end{definition}
En la definción anterior, que $\dot \phi$ no se anule en $\tilde I$ nos asegura que $\tilde{\vec c}$ también es una curva regular y que $\phi$ es monótona en $\tilde I$; es decir, para $\tilde\lambda \in \tilde I$ arbitraria se cumple que $\dot \phi(\tilde\lambda) > 0$ o $\phi(\tilde\lambda) < 0$.

Ahora enunciaremos y demostraremos, por razones estéticas, la invariancia de $\mathscr L(\vec c)$ bajo reparametrizaciones.
\begin{proposition}[Invariancia de la longitud de arco]
	Sea $\tilde{\vec c} = \vec c \circ \phi$ una reparametrización de $\vec c$. Entonces
	\begin{equation}
		\mathscr L (\tilde{\vec c}) = \mathscr L (\vec c).
	\end{equation}
\end{proposition}
\begin{proof}
	Sea $\phi : [\tilde a, \tilde b] \to [a, b]$ y consideremos primero el caso $\phi(\tilde a) = a$, $\phi(\tilde b) = b$ y $\phi$ estrictamente creciente. Entonces
	\begin{equation}
		\begin{split}
		\mathscr L(\tilde {\vec c}) &= \int_{\tilde a}^{\tilde b}\norm[\big]{\dot{\tilde{\vec c}}(\tilde \lambda)} \, d\tilde{\lambda}
		= \int_{\tilde a}^{\tilde b}\norm*{\frac{d}{d\tilde \lambda}{\vec c}\bigl( \phi(\tilde \lambda)\bigr)} \, d\tilde{\lambda}\\
		&= \int_{\tilde a}^{\tilde b}\norm[\big]{\dot \phi(\tilde \lambda)\dot{{\vec c}}\bigl( \phi(\tilde \lambda)\bigr)} \, d\tilde{\lambda}
		= \int_{\tilde a}^{\tilde b} \abs{\dot \phi(\tilde \lambda)}\norm[\big]{\dot{{\vec c}}\bigl( \phi(\tilde \lambda)\bigr)} \, d\tilde{\lambda}\\
		&= \int_{\tilde a}^{\tilde b}  \norm[\big]{\dot{{\vec c}}\bigl( \phi(\tilde \lambda)\bigr)} \dot \phi(\tilde \lambda)\, d\tilde{\lambda}
		= \int_{\phi(\tilde a)}^{\phi(\tilde b)}  \norm[\big]{\dot{{\vec c}} ( \lambda )}\, d\lambda\\
		&= \int_{a}^{b} \norm[\big]{\dot{{\vec c}} ( \lambda )}\, d\lambda
		= \mathscr L(\vec c).
		\end{split}
	\end{equation}
	Si en cambio $\phi(\tilde a) = b$, $\phi(\tilde b) = a$ y $\phi$ es estrictamente decreciente,
	\begin{equation}
		\begin{split}
		\mathscr L(\tilde {\vec c}) &= \int_{\tilde a}^{\tilde b} \abs{\dot \phi(\tilde \lambda)}\norm[\big]{\dot{{\vec c}}\bigl( \phi(\tilde \lambda)\bigr)} \, d\tilde{\lambda}
		= -\int_{\tilde a}^{\tilde b}  \norm[\big]{\dot{{\vec c}}\bigl( \phi(\tilde \lambda)\bigr)} \dot \phi(\tilde \lambda)\, d\tilde{\lambda}\\
		&= -\int_{\phi(\tilde a)}^{\phi(\tilde b)}  \norm[\big]{\dot{{\vec c}} ( \lambda )}\, d\lambda
		= -\int_{b}^{a} \norm[\big]{\dot{{\vec c}} ( \lambda )}\, d\lambda\\
		&= \int_{a}^{b} \norm[\big]{\dot{{\vec c}} ( \lambda )}\, d\lambda
		= \mathscr L(\vec c).
		\end{split}
	\end{equation}
\end{proof}

Los conceptos hasta ahora mencionados son aplicables para una clase muy general de curvas; pero para hablar de curvas ``pegadas'' a una superficie, necesitamos hallar las condiciones bajo las cuales estas curvas son regulares. Y más importantemente, necesitamos definir qué es una superficie. Esta última condición la relegaremos por el momento, recordando las herramientas útiles del análisis multivariable e introduciendo notación adicional.

\subsection{Conceptos Centrales de Cálculo Multivariable}
Para estudiar curvas en $\R^n$ la extensión del cálculo de una variable a cálculo vectorial fue, cuando menos sencilla, intuitiva; pero para tratar con superficies en $\R^n$ es necesario presentar varios conceptos claves de las funciones vectoriales $\vec f : U \subset \R^m \to \R^n$, donde $U$ es un subconjunto abierto de $\R^m$.

\begin{definition}
	La derivada parcial de $\vec f$ respecto a la variable $x^\mu, \mu = \overline{1, m},$ en $\vec p \in U$ se define como
	\begin{equation}
		\partialD{ \vec f}{x^\mu}(\vec p) \coloneqq \lim_{h \to 0} \frac{\vec f(\vec p + h\univ{\mu}) - \vec f(\vec p)}{h},
	\end{equation}
	donde $\univ{\mu}$ es el $\mu$-ésimo vector base de la base canónica de $\R^m$.
\end{definition}
Otras notaciones para la derivada parcial de $\vec f$ son
\begin{equation}
	\vec f_{x^\mu}, \qquad \partial_\mu \vec f, \qquad \vec f,_{\mu}.
\end{equation}

Las derivadas parciales, como su nombre lo indica, no son ``completas'' ni generalizan el concepto de \emph{diferenciabilidad} a funciones con entradas y valores vectoriales. En su lugar, empleamos la siguiente definición.
\begin{definition}
	Sea $\vec f : U \subset \R^m \to \R^n$ y $\vec p \in U$. Se dice que $\vec f$ es diferenciable en $\vec p$ si existe una transformación lineal única $D\vec f(\vec p) : \R^m \to \R^n$ tal que
	\begin{equation}
		\lim_{\vec h \to \vec 0}\frac{\norm{\vec f(\vec p + \vec h) - \vec f(\vec p) - D\vec f(\vec p)\vec h}}{\norm{\vec h}} = 0.
	\end{equation}
\end{definition}

Del álgebra lineal, a la transformación lineal $D\vec f(\vec p)$ se le puede asociar una matriz (respecto a las bases canónica de $\R^m$ y $\R^n$), la cual está dada por el siguiente resultado.
\begin{proposition}[La matriz jacobiana]\label{prop: jacobian}
	Si $\vec f: \R^m \to \R^n$ es diferenciable en $\vec p$, entonces sus derivadas parciales $\vec f_{,\mu}$ están bien definidas en $\vec p$ y
	\begin{equation}
		\mathcal M(D\vec f(\vec p)) = \Matrix{
			f\indices{^1_{,1}}(\vec p) & \dotsb & f\indices{^1_{,m}}(\vec p)\\
			\vdots & & \vdots\\
			f\indices{^n_{,1}}(\vec p) & \dotsb & f\indices{^n_{,m}}(\vec p)
		}.
	\end{equation}
	La notación se suele abreviar denotando simplemente a $\mathcal M(D\vec f(\vec p))$  como $D\vec f(\vec p) $.
\end{proposition}
En la proposición anterior se ilustra el beneficio de la notación empleada.
Notése la separación horizontal que hay entre los subíndices y los superíndices.
\begin{remark}
	Para el caso $m = 1$, la representación matricial de $D\vec f(\vec p)$ se reduce a
	\begin{equation}
		D\vec f(\lambda) = \Matrix{\dot f^1(\lambda) \\ \vdots \\ \dot f^n(\lambda)}.
	\end{equation}
\end{remark}

En cálculo multivaribale, así como en cálculo de una sola variable, se pueden componer funciones y esas funciones compuestas pueden diferenciarse si cada una es diferenciable por separado. Más precisamente, tenemos el siguiente teorema.
\begin{theorem}[Regla de la Cadena]\label{teo: chain}
	Sean $\vec f : V \subset \R^m \to \R^n$ y $\vec g : U \subset \R^l \to \R^m$ tal que $\vec g$ es diferenciable en $\vec p$ y $\vec f$ es diferenciable en $\vec g(\vec p)$. Entonces $\vec f \circ \vec g$ es diferenciable en $\vec p$ y
	\begin{equation}
		D(\vec f \circ \vec g)(\vec p) = D\vec f(\vec g(\vec p)) D\vec g(\vec p).
	\end{equation}
\end{theorem}

Así como una curva puede derivarse múltiples veces, las funciones vectoriales pueden derivarse parcialmente varias veces y respecto a diferentes variables: por ejemplo, la derivada parcial de $\vec f_{,\mu}$ respecto a $x^\nu$ es
\begin{equation}
	\vec f_{,\mu,\nu} \coloneqq \frac{\partial^2 \vec f}{\partial x^\nu \partial x^\mu}.
\end{equation}
Al conjunto de funciones $\vec f : U \subset \R^m \to \R^n$ cuyas $r$-ésimas derivadas parciales existen y son continuas en $U$ se denota por $C^r(U; \R^n)$. El conjunto $C^\infty(U; \R^n)$ es el conjunto de todas las funciones cuyas derivadas parciales de todos los órdenes existen y son continuas, y si $\vec f \in C^\infty(U; \R^n)$ decimos que $\vec f$ es suave en $U$.

Un resultado importante del análisis es que si una función $\vec f$ pertenece a $C^2(U; \R^n)$, entonces sus derivadas parciales conmutan:
\begin{equation}
	\vec f_{,\mu, \nu} = \vec f_{,\nu,\mu}.
\end{equation}

El siguiente resultado nos asegurará la suavidad de curvas en superficies.
\begin{proposition}\label{prop: compSmooth}
	Sean $V \subset \R^m$, $f \in C^\infty(V; \R^n)$ y $U \subset \R^l$, $\vec g \in C^\infty(U; \R^m)$ tal que $\vec g(U) \subset V$, entonces la función compuesta $\vec f \circ \vec g : U \to \R^n$ pertenece a $C^\infty(U; \R^n)$.
\end{proposition}
Mientras que la siguiente definición nos indicará como calcular la velocidad de la curva en la superficie.
\begin{definition}\label{def: directional}
	Sea $\vec f: U \subset \R^m \to \R^n$, $\vec p \in U$ y $\vec v \in \R^m$. La derivada direccional de $\vec f$ en $\vec p$ en la dirección $\vec v$ es
	\begin{equation}
		D_{\vec v} \vec f(\vec p) \coloneqq \lim_{h \to 0} \frac{\vec f(\vec p + h\vec v) - \vec f(\vec p)}{h} = \frac{d}{d\lambda}(\vec f \circ \vec c)(0),
	\end{equation}
	donde $\vec c(\lambda) = \vec p + \lambda \vec v$
\end{definition}

\begin{proposition}
	En la definción anterior, si $\vec f$ es diferenciable en $\vec p$, entonces
	\begin{equation}
		D_{\vec v} \vec f(\vec p) = D\vec f(\vec p)\vec v.
	\end{equation}
\end{proposition}

\begin{proposition}
	Podemos extender la definición \ref{def: directional} a curvas diferenciables. Sea $\vec f$ como en la definición \ref{def: directional} y $\vec c : I \to U$ una curva diferenciable en $\lambda \in I$ tal que $\vec c(\lambda) = \vec p$ y $\dot{\vec c}(\lambda) = \vec v$ . Entonces
	\begin{equation}
		D_{\vec v} \vec f(\vec p) = \frac{d}{d\lambda}(\vec f \circ \vec c)(\lambda).
	\end{equation}
	Si $\vec f$ es diferenciable en $\vec p$, entonces
	\begin{equation}
		D_{\vec v} \vec f(\vec p) = D\vec f(\vec c(\lambda))\dot{\vec c}(\lambda).
	\end{equation}
\end{proposition}

\begin{remark}
	Para emplear la proposición \ref{prop: jacobian}, hacemos la siguiente observación
	\begin{equation}
		\vec f_{,_\mu}(\vec p) = \Matrix{
			f\indices{^1_{,\mu}}(\vec p) \\
			\vdots \\
			f\indices{^n_{,\mu}}(\vec p).
		}.
	\end{equation}
	Entonces,
	\begin{equation}
		\frac{d}{d\lambda}(\vec f \circ \vec c)(\lambda) = \sum_{\mu=1}^m \dot c^\mu(\lambda)\vec f_{,_\mu}(\vec c(\lambda)).
	\end{equation}
	Empleando el convenio de la suma de Einstein, donde índices repetidos \emph{una sola vez} arriba y abajo indican una suma sobre el rango entendido por el contexto, podemos escribir (omitiendo el parámetro $\lambda$)
	\begin{equation}
		\frac{d}{d\lambda}(\vec f \circ \vec c) = \dot c^\mu \vec f_{,_\mu}(\vec c).
	\end{equation}
\end{remark}

Finalmente, enunciamos un teorema crítico para la geometría diferencial.
\begin{theorem}[Teorema de la Función Inversa]
	Sean $U\subset \R^n$ abierto, $\vec f \in C^1(U; \R^n)$ y $\vec p \in U$ tal que $D\vec f(\vec p)$ es invertible. Entonces existe una vecindad $V \subset U$ de $\vec p$ tal que $\vec f$ tiene una función inversa diferenciable. De forma más precisa, existen vecindades $V$ de $\vec p$ y $W$ de $\vec f(\vec p)$, y una función $C^1(W; V)$ tal que
	\begin{equation}
		\vec f(\vec g(\vec q)) = \vec q \quad \forall \vec q \in W, \qquad \vec g(\vec f(\vec r)) = \vec r \quad \forall \vec r \in V.
	\end{equation}
	Además,
	\begin{equation}
		D\vec g(\vec f(\r)) = \bigl[D\vec f(\r) \bigr]^{-1}.
	\end{equation}
\end{theorem}
\begin{proposition}
	Si $\vec f$ es suave, entonces su función inversa garantizada por el teorema anterior también es suave.
\end{proposition}

Para los casos donde $\vec f$ esté definida sobre un conjunto $U$ no abiertos, hacemos las siguiente definición.
\begin{definition}
	Sea $X \subset \R^m$ un conjunto no necesariamente abierto. Se dice que $\vec f: X \to \R^n$  es suave si para toda $\vec p \in X$ existe una vecindad $U \subset \R^m$ suya y una función $\tilde{\vec f} \in C^\infty(U; \R^n)$ tal que $\tilde{\vec f}\rvert_{X\cap U} = {\vec f}$.
\end{definition}

Para poder pasar a la geometría diferencial que nos interesa, hacemos las últimas definiciones fundamentales.
\begin{definition}
	Sea $U \subset \R^n$. Un conjunto $V \subset U$ se dice que es \emph{abierto en} $U$ si $V$ es la intersección de $U$ con un subconjunto abierto de $\R^n$.
	
	Si $\vec p \in U$, \emph{una vecindad de} $\vec p$ \emph{en} $U$ es un subconjunto de $U$ que es abierto en $U$ y que contiene a $\vec p$.
\end{definition}

\begin{definition}
	Dos subconjunto $X \subset \R^m$ y $Y\subset \R^n$ son \emph{difeomórficos} si existe una función invertible suave $\vec f: X \to Y$ tal que su inversa también sea suave.  A tal función se le conoce como un \emph{difeomorfismo}.
\end{definition}

\section{Superficies en el Espacio}
La ausencia de evidencia no implica una evidencia de ausencia. En este principio se basan los argumentos más empleados para combatir a los terraplanistas cuando ellos afirman que vivimos en un mundo plano, basándose en que no se puede percibir en el día a día la curvatura de la Tierra que asegura el modelo de ella más aceptado, el esférico.

Para comenzar a dotar de precisión a esta discusión recurrimos a los mapas. Un mapa \textit{plano} de una región como una ciudad, un estado o incluso un país nos permite hacer una correspondencia entre nuestra posición y las dos coordenadas de latitud y longitud. Es decir, relacionamos nuestra posición en un mundo tridimensional con un punto en un plano. De manera concisa, decimos que localmente (en una porción muy pequeña de la Tierra en donde nos encontramos) el globo terráqueo es euclidiano.

Esta es la idea central detrás de la geometría diferencial de superficies bidimensionales, que la extiende de la superficie de la Tierra a superficies más generales. Empezamos entonces con la siguiente definición de los conjuntos de interés.

\begin{definition}\label{def: surface}
	Un conjunto $\surface \subset \R^3$ es una \emph{superficie regular} si cada uno de sus puntos tiene una vecindad $V$ en $\surface$ que es difeomórfica a un conjunto abierto $U \subset \R^2$.
	
	Al difeomorfismo $\patch : U \to V$ se le conoce como un \emph{parche coordenado}, y a una colección de parches coordenados que cubre a todo $\surface$ se le llama un \emph{atlas de} $\surface$.
\end{definition}

Una propiedad de gran utilidad para poder emplear el lenguaje de álegbra lineal es el siguiente.
\begin{proposition}
	Sea $\patch$ un parche coordenado de $\surface$. Para toda $\vec q \in U$ la transformación lineal $D\patch(\vec q) \in \mathcal L(\R^2; \R^3)$ tiene rango dos.
\end{proposition}
Esto quiere decir que el espacio generado por los vectores $\patch_{,1}(\vec q), \patch_{,2}(\vec q)$ es bidimensional.
\begin{equation}
	\dim \Span(\patch_{,1}(\vec q), \patch_{,2}(\vec q)) = 2.
\end{equation}
Esto a su vez implica que los vectores $\patch_{,1}(\vec q), \patch_{,2}(\vec q)$ son linealmente independientes y forman una base de un subespacio bidimensional de $\R^3$: un plano.

Para distinguir las coordenadas empleadas en los conjuntos $U \subset \R^2$ de las coordenadas cartesianas $x, y, z$ del espacio, denotaremos a las primeras por $q^1, q^2$ y las llamaremos coordenadas locales en $V$.

Incluimos ahora las curvas pegadas a las superficies que habíamos mencionado anteriormente.
\begin{definition}
	Sea $\surface$ una superficie regular. Una \emph{curva regular en} $\surface$ es una curva regular en $\R^3$ cuya traza está contenida en $\surface$.
	
	El \emph{plano tangente} $T_{\vec p}\surface$ de $\surface$ en el punto $\vec p \in \surface$ es el conjunto de todas las velocidades iniciales de las curvas regulares en $\surface$ con posición inicial $\vec p$:
	\begin{equation}
		T_{\vec p}\surface \coloneqq \set{\dot{\vec c}(0) \ \rvert \ \mathrm{trace}(\vec c) \subset \surface, \vec c(0) = \vec p}.
	\end{equation}
\end{definition}

La definición anterior para el plano tangente tiene la ventaja que es independiente del parche elegido para $\surface$; y una vez que se tiene un parche de $\surface$, se puede determinar fácilmente a $T_{\vec p}\surface$ con el siguiente resultado.
\begin{lemma}
	El conjunto $T_{\vec p}\surface$ de $\surface$ es un subespacio bidimensional de $\R^3$ y
	\begin{equation}
		T_{\vec p}\surface = \Span(\patch_{,1}(\vec q), \patch_{,2}(\vec q)).
	\end{equation}
\end{lemma}

Para facilitar la construcción de curvas regulares en una superficie, que son los objetos de interés en este trabajo, aprovechamos la regla de la cadena de la siguiente forma.
\begin{proposition}
	Sea $\vec c : I \to U$ una curva regular y $\patch: U \to \surface$ un parche de $\surface$. La curva definida por
	\begin{equation}\label{eq: gcurve}
		\bgamma(\lambda) \coloneqq (\patch \circ \vec c)(\lambda),
	\end{equation}
	es regular en $\surface$.
\end{proposition}
\begin{proof}
	Como la imagen de $\patch$ está en $\surface$, es evidente que la traza de $\bgamma$ está en $\surface$. Por la proposición \ref{prop: compSmooth}, $\patch \circ \vec c$ es suave y de la regla de la cadena
	\begin{equation}
		\dot \bgamma = \dot c^\mu \patch_{,\mu}(\vec c),
	\end{equation}
	como $\patch_{,\mu}$ son linealmente independientes, $\dot \bgamma$ solo se anularía si $\dot c^\mu = 0$ para $\mu = 1, 2$. Pero como $\vec c$ es regular, $\dot c^\mu$ siempre son diferentes de cero.
\end{proof}

\subsection{La Métrica Riemanniana}
La idea de \emph{distancia en una superficie} que ocuparemos está definida a través de la longitud de arco de una curva regular en la superficie. Recordemos de álgebra lineal que al espacio vectorial $\R^n$ lo podemos dotar del producto interno estándar euclidiano $\innerp{}{}$ definido como
\begin{equation}
	\innerp{(x^1, \dotsc, x^n)}{(y^1, \dotsc, y^n)} \coloneqq \sum_{\mu=1}^n x^\mu y^\mu.
\end{equation}
Y que la norma que induce es la norma estándar euclidiana
\begin{equation}
	\norm{(x^1, \dotsc, x^n)} = \sqrt{\innerp{(x^1, \dotsc, x^n)}{(x^1, \dotsc, x^n)} } \,.
\end{equation}
Definimos, aparentemente fortuita por el momento, el concepto de métrica.
\begin{definition}
	La métrica $g : \R^n \times \R^n \to \R$ es la función bilineal simétrica definida por
	\begin{equation}
		g(\vec u, \vec v) \coloneqq \innerp{\vec u}{\vec v}.
	\end{equation}
\end{definition}

Sin embargo, cuando se emplean bases $\{\vec e_\mu\}_{\mu=1}^n$ más generales de $\R^n$ se vuelve necesario introducir la matriz de la métrica $g$ respecto a la base $\{\vec e_\mu\}_{\mu=1}^n$.
\begin{definition}
	La matriz $\mathcal M(g, \{\vec e_\mu\}_{\mu=1}^n)$ de la métrica respecto a la base $\{\vec e_\mu\}_{\mu=1}^n$ es la matriz cuadrada $n \times n$ cuyos elementos son
	\begin{equation}
		g_{\mu\nu} \coloneqq g(\vec e_\mu, \vec e_\nu).
	\end{equation}
\end{definition}
\begin{remark}
	En la base estándar euclidiana, la matriz de $g$ es la delta de Kroenecker.
	\begin{equation}
		g_{\mu\nu} = g(\univ \mu, \univ \nu) = \innerp{\univ \mu}{\univ \nu} = \delta_{\mu\nu}.
	\end{equation}
\end{remark}
\begin{remark}
	Haber definido la matriz de la métrica nos permite aprovechar la convención de la suma de Einstein. Sea $\{\vec e_\mu\}_{\mu=1}^n$ una base de $\R^n$ y $\vec u = u^\mu \vec e_\mu$ y $\vec v = v^\nu \vec e_\nu$. Entonces
	\begin{equation}
		\innerp{\vec u}{\vec v} = u^\mu v^\nu \innerp{\vec e_\mu}{\vec e_\nu} = g_{\mu\nu}u^\mu v^\nu .
	\end{equation}
\end{remark}

De manera ilustrativa calculamos el producto interno entre las velocidades de dos curvas regulares en $\surface$, $\bgamma_1, \bgamma_2 : I \to \surface$ de la forma \eqref{eq: gcurve}, tales que para alguna $\xi \in I$, $\vec c_1(\xi) = \vec c_2(\xi) = \vec q$.
\begin{equation}
	\innerp{\dot\bgamma_1(\xi)}{\dot \bgamma_2(\xi)} = \dot c_1^\mu(\xi) \dot c_2^\nu(\xi) g(\patch_{,\mu}(\vec c_1(\xi)), \patch_{,\nu}(\vec c_2(\xi)))
	= \dot c_1^\mu(\xi) \dot c_2^\nu(\xi) g_{\mu\nu}(\vec q).
\end{equation}
Vemos entonces que el producto interno entre las dos curvas no depende solo de los valores de sus componentes, sino que también depende del valor de la métrica en cada punto de $\surface$.

Esto nos lleva a la siguiente definición.
\begin{definition}
	Una métrica riemmaniana en una superficie regular es una regla de correspondencia que asocia a cada punto $\vec p \in \surface$ un producto interno $\innerp{}{}_{\vec p}$ \emph{diferenciable} en el espacio tangente $T_{\vec p}\surface$.
	
	Una superficie regular con una métrica riemanniana se le conoce como una \emph{superficie riemanniana}.
\end{definition}
En la definición anterior, que un producto interno $\innerp{}{}_{\vec p}$ sea diferenciable signfica que existe un parche $\patch : U \to \surface$ alrededor de $\vec p$ tal que
\begin{equation}
	\innerp{\patch_{,\mu}(\vec q)}{\patch_{,\nu}(\vec q)}_{\vec p} = g_{\mu\nu}(\vec q)
\end{equation}
es diferenciable en $U$. Cuando quedé claro que se está empleando el producto interno en $\vec p$ se omitirá este último.

Podemos ahora llegar a una expresión para la longitud de arco de una curva regular en $\surface$ en coordenadas locales.
\begin{remark}
	Sea $\bgamma$ una curva como en la proposición \ref{eq: gcurve}, con $I = [a,b]$. Entonces su longitud de arco puede calcularse como
	\begin{equation}\label{eq: surfArc}
		\mathscr L(\bgamma) = \int_{a}^{b}\sqrt{g(\dot \bgamma, \dot \bgamma)} \, d\lambda
		= \int_{a}^{b} \sqrt{
			g_{\mu\nu} \dot c^\mu \dot c^\nu
		}\, d\lambda.
	\end{equation}
\end{remark}

\subsection{Símbolos de Christoffel}
Como se estableció anteriormente, un parche $\patch$ de una superficie regular $\surface$ induce en cada punto $\vec p \in \surface$ una base del subespacio $T_{\vec p}\surface$. Del álgebra lineal sabemos que si introducimos un tercer vector linealmente independiente a $\patch_{,\mu}$ podemos generar una base de $\R^3$. Esto nos conduce a hacer la siguiente definición.
\begin{definition}
	Un \emph{vector normal} $\vec N \in \R^3$ en $\vec p$ a $\surface$ es un vector ortogonal a $T_\vec p\surface$.
\end{definition}
Entonces, la lista $\{\patch_{,1}(\vec q), \patch_{,2}(\vec q), \vec N(\vec q)\}$ es linealmente independiente en $\R^3$ y por lo tanto una base de $\R^3$.
Por las propiedades del producto cruz en $\R^3$, la siguiente definición es la de un vector normal unitario a $\surface$ en $\patch(\vec q)$:
\begin{equation}
	\vec N(\patch(\vec q)) \coloneqq \frac{\patch_{,1}(\vec q) \times \patch_{,2}(\vec q)}{\norm{\patch_{,1}(\vec q) \times \patch_{,2}(\vec q)}}.
\end{equation}

La definición de métrica riemanniana nos conduce a buscar una expresión para $D g_{\alpha\beta}$. Determinamos entonces una expresión para las derivadas parciales de las componentes de la métrica.
\begin{equation}
	g_{\alpha\beta, \mu} = \partial_\mu \innerp{\patch_{,\alpha}}{\patch_{,\beta}}_{\vec p}
	= \innerp{\patch_{,\alpha}}{\patch_{,\beta,\mu}}_{\vec p} + \innerp{\patch_{,\alpha, \mu}}{\patch_{,\beta}}_{\vec p}.
\end{equation}

Para poder expresar a las segundas derivadas parciales de $\patch$ con la base intrínsica\\ $\{\patch_{,1}(\vec q), \patch_{,2}(\vec q), \vec N(\vec q)\}$ a $\surface$ es necesario introducir los símbolos de Christoffel.
\begin{definition}
	Los símbolos de Christoffel $\ChrisSym{\alpha}{\mu\nu} : U \to \R$, donde $\alpha, \mu,\nu = \overline{1,2}$, se definen como las funciones que en cada punto $\vec p \in \surface$ satisfacen.
	\begin{equation}
		\patch_{,\mu,\nu} = \ChrisSym{\alpha}{\mu\nu} \patch_{,\alpha} + \innerp{\patch_{,\mu,\nu}}{\vec N}_\vec{p}\vec N.
	\end{equation}
	Por la conmutatividad de las derivadas parciales, los símbolos de Christoffel son simétricos en los subíndices:
	\begin{equation}
		\ChrisSym{\alpha}{\nu\mu} = \ChrisSym{\alpha}{\mu\nu}.
	\end{equation}
\end{definition}
\begin{definition}[Derivada Covariante]
	Sea $\surface$ superficie regular y $\vec c: I \to \surface$ una curva regular en $\surface$. Si $\vec v : I \to \R^3$ es una función suave tal que $\vec v(\lambda) \in T_{\vec c(\lambda)}\surface$ para toda $\lambda \in I$, llamado un campo vectorial sobre $\vec c$, entonces se define la \emph{derivada covariante} de $\vec v$: $\vec v_{;\lambda}$ es el campo vectorial sobre $\vec c$ que satisface
	\begin{equation}
		\vec v_{;\lambda} \coloneqq P_{\mathcal T} \dot{\vec v}(\lambda), \qquad \lambda \in I, \quad \mathcal T = T_{\vec c(\lambda)}\surface,
	\end{equation}
	donde $P_{\mathcal T} : \R^3 \to T_{\vec c(\lambda)}\surface $ es el operador de proyección ortogonal sobre $T_{\vec c(\lambda)}\surface$.
\end{definition}

\begin{remark}
	De la definición de los símbolos de Christoffel,
	\begin{equation}\label{eq: covariant}
		\patch_{,\mu;\nu} = \ChrisSym{\alpha}{\mu\nu} \patch_{,\alpha}.
	\end{equation}
	Y se cumple la relación de simetría
	\begin{equation}
		\patch_{,\mu;\nu} = \patch_{,\nu;\mu}.
	\end{equation}
\end{remark}

En este punto hemos adquirido todas las herramientas de geometría diferencial necesarias para nuestro enfoque de las geodésicas. Y aunque contemos con la expresión \eqref{eq: surfArc} para calcular la longitud de arco de una curva en una superficie, esto no es lo mismo que calcular la \emph{distancia} entre dos puntos de $\surface$.

\section{Distancia en una Superficie}

Durante la siguiente subsección nos olvidaremos de las superficies y curvas regulares que hemos estado formando, y nos enfocaremos brevemente en el concepto más general y abstracto de distancia en un conjunto igualmente general y abstracto.

\begin{definition}[Espacio Métrico]
	Sea $M$ un conjunto no vacío y $p, q, r \in M$. Un \emph{espacio métrico} (e. m.) es un par $(M, d)$ donde $d: M\times M \to \R$ es una función que satisface:
	\begin{enumerate}[(i)]
		\item Positividad: $d(p, q) \ge 0$;
		\item Definitivdad: $d(p, q) = 0$ sii $p = q$;
		\item Simetría: $d(p, q) = d(q, p)$;
		\item Desigualdad del triángulo: $d(p, q) \le d(p, r) + d(r, q)$.
	\end{enumerate}
\end{definition}
Por ejemplo, la distancia inducida por la norma $\norm{}$ como $d(\vec u, \vec v) = \norm{\vec u - \vec v}$ satisface las cuatro propiedades anteriores.

De la extensa teoría de los espacios métricos, nosotros extraeremos los siguientes resultados de \parencite{clapp-2010}.

\begin{definition}
	Sea $(M, d)$ un e. m. y $p, q \in M$. Una \emph{trayectoria de} $p$ \emph{a} $q$ \emph{en} $M$ es una función continua $\gamma : [0, 1] \to M$ tal que $\gamma(0) = p$ y $\gamma(1) = q$.
	
	Su longitud se define como
	\begin{equation}
		\mathscr L(\gamma) \coloneqq \sup_{m \in \mathbb N}\left\{ \sum_{k=1}^m d(\gamma(\lambda_{k-1}), \gamma(\lambda_k)) \  \biggr\rvert \ 0 = \lambda_0 \le \lambda_1 \le \dotsb \le \lambda_m = 1\right\}.
	\end{equation}
\end{definition}

Podemos notar que si $\mathscr L(\gamma) < \infty$, por la unicidad del supremo, se define una función\\ $C^0([0, 1]; M) \to \R$. De igual forma, si el conjunto en la definición no es acotado extendemos a la función longitud a $\mathscr L : C^0([a, b]; M) \to \R\cup\{\infty\}$. Notemos que estamos definiendo trayectorias sobre el intervalo $[0,1]$ no y sobre uno más general $[a,b]$. Esto porque la invariancia de la longitud de arco de una curva también se cumple para reparametrizaciones de una trayectoria

Así como para las curvas regulares en $\R^n$, podemos definir qué es una trayectoria parametrizada por longitud de arco.
\begin{definition}\label{def: paramArc}
	Una trayectoria $\gamma \in C^0([0, 1], M)$ de longitud finita está parametrizada por longitud de arco si, para toda $\lambda \in [0, 1]$,
	\begin{equation}
		\mathscr L(\gamma; 0, \lambda) = \mathscr L(\gamma)\lambda.
	\end{equation}
\end{definition}

Tomamos ahora como espacio al conjunto de trayectorias de la siguiente manera.
\begin{definition}\label{def: tray}
	El \emph{espacio (normado) de trayectorias} $(\mathcal T_{p, q}, d_\infty)$ se define como
	\begin{equation}
		\mathcal T_{p, q}(M) \coloneqq \{ \gamma \in C^0([0, 1]; M) \ \vert \  \gamma(0) = p, \gamma(1) = q\},
	\end{equation}
	con la distancia del supremo
	\begin{equation}
		d_\infty(\gamma, \tau) \coloneqq \max_{\lambda \in [0,1]} d(\gamma(\lambda), \tau(\lambda)),
	\end{equation}
	y la función longitud $\mathscr L : \mathcal T_{p, q}(M) \to \R \cup \{\infty\}$.
\end{definition}

\parencite{clapp-2010} muestra que la definición \ref{def: paramArc} no es fortuita, pues, como veremos más adelante, la existencia de trayectorias de longitud mínima requiere de la \emph{compacidad} de $M$; que $M$ sea compacto implica que toda sucesión de elementos $(p_k) \in M^{\mathbb N}$ posee una subsucesión que converge en $M$. Sin embargo, admitir múltiples parametrizaciones de una trayectoria impide la compacidad.
Definimos entonces el conjunto $\hat{\mathcal T}_{p,q}(M)$.
\begin{equation}
	\hat{\mathcal T}_{p,q}(M) \coloneqq \{ \gamma \in \mathcal T_{p, q}(M) \ \vert \  \mathscr(\gamma) < \infty, \mathscr L(\gamma, 0; \lambda) = \mathscr L(\gamma)\lambda\}.
\end{equation}
Para curvas regulares en $\R^n$ es posible siempre, aunque no de manera explícita, encontrar una reparametrización por longitud de arco. Este resultado también aplica para trayectorias en espacios métricos arbitrarios.
\begin{lemma}
	Para cada $\gamma \in {\mathcal T}_{p,q}(M)$ de longitud finita existe $\hat \gamma \in \hat{\mathcal T}_{p,q}(M)$ tal que $\mathscr L(\gamma) = \mathscr L(\hat \gamma)$.
\end{lemma}

Antes de enunciar el resultado de \parencite{clapp-2010} crucial para este trabajo, hacemos la definición central a este escrito.

\begin{definition}[Trayectoria Geodésica]
	Sea $M$ un espacio métrico y $p, q \in M$. Una trayectoria geodésica $\hat{\gamma}$ es un elemento de $\mathcal T_{p, q}(M)$ que cumple $\mathscr L(\hat \gamma) \le \mathscr L( \gamma)$ para toda $\gamma \in \mathcal T_{p, q}(M)$.
\end{definition}
El siguiente teorema nos da entonces las condiciones necesarias y suficientes para encontrar una trayectoria geodésica.
 
\begin{theorem}[Existencia de trayectorias geodésicas]
	Sea $M$ espacio métrico compacto y $p, q \in M$. Si el conjunto $\mathcal T_{p, q}(M)$ es no vacío, entonces existe una trayectoria de longitud mínima en $\mathcal T_{p, q}(M)$.
\end{theorem}

Este teorema transforma entonces al problema inicial de encontrar una geodésica en una superficie $\surface$ de una cuestión geométrica a una cuestión de demostrar que $\surface$ es compacta. Nos enfocamos ahora nuevamente en las superficies regulares $\surface$ y sus propiedades.

\subsection{La Distancia Intrínseca}
El primer paso para demostrar que $\surface$ es compacta es recordar que un subconjunto de $\R^n$ es compacto sii es cerrado y acotado. Esta propiedad de $\surface$ está asegurada, como mostraremos, por el difeomorfismo $\patch$ en su definición.

\begin{proposition}
	Sean $M$ y $X$ espacios métricos y $\phi : M \to X$ una función. Son equivalentes
	\begin{enumerate}[(i)]
		\item La función $\phi$ es continua;
		\item $\phi^{-1}(U)$ es abierto en $M$ para todo subconjunto abierto $U$ de $X$;
		\item  $\phi^{-1}(C)$ es cerrado en $M$ para todo subconjunto cerrado $C$ de $X$.
	\end{enumerate}
\end{proposition}

\begin{proposition}
	Si $\phi : M \to X$ es continua y $K$ es un subconjunto compacto de $M$, entonces $\phi(K)$ es un subconjunto compacto de $X$.
\end{proposition}

Podemos entonces enfocar la cuestión de demostrar que un subconjunto de $\surface$ es compacto a demostrar que un subconjunto cerrado de $U$ en la definición \ref{def: surface} es compacto. Es decir, para dos puntos $\vec p, \vec q \in \surface$, debemos demostrar que un subconjunto $K$ de $U$ que contenga a $\patch^{-1}(\vec p)$ y $\patch^{-1}(\vec q)$ es cerrado y acotado, pues al ser $\patch$ un difeomorfismo automáticamente es continua.

La segunda condición para la existencia de una geodésica en $\surface$ que una a $\vec p$ y $\vec q$ es asegurar que $\mathcal T_{\vec p, \vec q}(\surface)$ es no vacío. Esto lo comenzamos a hacer a través de la siguiente definición.
\begin{definition}[Arcoconexidad]
	Un subconjunto $N$ de un espacio métrico $M$ es arcoconexo si para cualquier par de puntos $p, q \in N$ el conjunto $\mathcal T_{p, q}(N)$ es no vacío.
\end{definition}

\begin{definition}[Conexidad]
	Un conjunto $D$ es desconexo si existen dos conjunto abiertos $A$ y $B$ disjuntos tales que $D \subset A \cup B$ pero $D \not\subset A$ y $D \not\subset B$. Se dice que $D$ es conexo si no es desconexo.
\end{definition}

La arcoconexidad de un conjunto asegura la conexidad del mismo; pero un conjunto conexo es arcoconexo solo si además es abierto.

\begin{proposition}
	Si $\phi : X \to Y$ es una transformación continua y $U$ es un subconjunto conexo de $X$, entonces $\phi(X)$ es conexo en $Y$.
\end{proposition}

Para una superficie regular y un difeomorfismo $\patch : U \to \surface$, la elección de un subconjunto $K \subset U$ está guiada por lo siguiente: si $K$ es abierto y conexo en $U$, entonces $\patch(K) \subset U$ es abierto y conexo en $\surface$ y por lo tanto será arcoconexo.
Sin embargo, queda el detalle de que $K$ sea cerrado para asegurar la compacidad de $\patch(K)$. Esto se resuelve simplemente tomando su cerradura $\overline K$.

\begin{proposition}
	Sean $\surface$ una superficie regular, $\vec p, \vec q \in \surface$ y $\patch : U \to \surface$, con $U \subset \R^2$ abierto y conexo, un parche tal que $\patch(U)$ contenga a $\vec p$ y $\vec q$. Entonces en $\patch(\overline U)$ existe una geodésica entre $\vec p$ y $\vec q$.
\end{proposition}

Conocemos entonces las condiciones necesarias para hacer la siguient definición.
\begin{definition}
	Sea $\surface$ una superficie regular arcoconexa. La \emph{distancia intrínseca} $d: \surface \times \surface \to \R$ de $\surface$ se define como
	\begin{equation}
		d(\vec p, \vec q) \coloneqq \inf  \{\mathscr L(\vec c) \ \vert \ \vec c \in \mathcal T_{\vec p, \vec q}\surface, \vec c \in C^1([0, 1]; \surface) \}.
	\end{equation}
\end{definition}
\begin{proposition}
	La función distancia intrínseca de $\surface$ es una distancia.
\end{proposition}
\begin{proof}
Mostraremos que $d$ satisface las cuatro propiedades de una distancia.
	\begin{enumerate}[(i)]
		\item Por definición $\mathscr L(\vec c) \ge 0$, por lo que $d(\vec p, \vec q) \ge 0$.
		\item Si $\vec p = \vec q$, podemos definir una curva regular $\vec c$ en $\surface$ como $\vec c(\lambda) \coloneqq \vec p$ para toda $\lambda \in [0,1]$. Entonces
		\begin{equation}
			\mathscr L(\vec c) = \int_0^1 \norm{\dot{ \vec c}} \, d\lambda = \int_0^1 (0) \, d\lambda = 0.
		\end{equation}
		Si $d(\vec p, \vec q) = 0$, entonces para cualquier curva $\vec c$ regular en $\surface$ que una a $\vec p$ y $\vec q$
		\begin{equation}
			 \int_0^1 \norm{\dot{ \vec c}} \, d\lambda  = 0.
		\end{equation}
		Pero la función $\norm{\dot{\vec c}}$ es siempre no negativa, por lo que $\norm{\dot{\vec c}} = 0$ para toda $\lambda \in [0,1]$. Luego la función $\vec c$ es constante y $\vec p = \vec c(0) = \vec q$.
		
		\item Para una curva geodésica $\vec c \in \mathcal T_{\vec p, \vec q}(\surface)$ elegimos la reparametrización $\phi(\tilde \lambda) = 1 - \tilde \lambda, \tilde \lambda \in [0,1]$. Entonces $\tilde{\vec c}(0) = \vec c(\phi(0)) = \vec c(1) = \vec q$ y $\tilde{\vec c}(1) = \vec c(\phi(1)) = \vec c(0) = \vec p$ y $\tilde{\vec c} \in   \mathcal T_{\vec q, \vec p}(\surface)$. Por la invariancia de la longitud de arco
		\begin{equation}
			d(\vec q, \vec p) = \mathscr L(\tilde{\vec c}) = \mathscr L(\vec c) = d(\vec p, \vec q).
		\end{equation}
		
		\item Sean $\vec c_1$ una geodésica que une a $\vec p$ y $\vec r$ y $\vec c_2$ una que une a $\r$ y $\vec q$. Entonces, eligiendo reparametrizaciones apropiadas de $\vec c_1, \vec c_2$ la curva
		\begin{equation}
			\vec c(\lambda) \coloneqq \begin{cases}
				\vec c_1(\lambda), & 0 \le \lambda < 1/2,\\
				\vec c_2(\lambda), & 1/2 \le \lambda \le 1,
			\end{cases}
		\end{equation}
		une a $\vec p$ con $\vec q$.
			Por definición, $d(\vec p, \vec q) \le \mathscr L(\vec c)$, y
	\begin{equation}
		\mathscr L(\vec c) = \mathscr L(\vec c_1) + \mathscr L(\vec c_2) = d(\vec p, \r) + d(\vec r, \vec q).
	\end{equation}
	Luego $d(\vec p, \vec q) \le d(\vec p, \r) + d(\vec r, \vec q)$.
	\end{enumerate}
	
	Por lo tanto $(\surface, d)$ es espacio métrico.
\end{proof}

\section{La Ecuación de la Geodésica}
Presentamos ahora el método para encontrar las curvas geodésicas tal cómo han sido planteadas.

En las secciones anteriores hemos colocado sigilosamente algunos elementos: definimos en \ref{def: arcL} la longitud de arco como una función que asocia a cada curva $\vec c$ regular un número no negativo. En la definición \ref{def: tray} equipamos al espacio $\mathcal T_{p, q}(M)$ una distancia.

Esto son los elementos necesarios para introducir el \emph{cálculo de variaciones}, donde el funcional que buscamos minimizar es la longitud de arco $\mathscr L (\vec c)$ de una curva regular en una superficie $\surface$.

El resultado central al cálculo variacional son las \emph{ecuaciones de Euler-Lagrange}, que determinan, como en cálculo de una variable, en qué funciones un funcional tiene un extremo, mas no aseguran que tal extremo sea un mínimo.

\begin{theorem}[Ecuaciones de Euler-Lagrange]
	Consideremos un espacio métrico $(M, d)$ y un funcional $S : U \to \R$ definido en un subconjunto abierto $X$ de $M$.
	Sea $\mathcal L : (\lambda, y^\alpha, z^\alpha), \alpha = \overline{1, n}$ una función de clase $C^2$. Una condición necesaria para que la curva $\vec c \in C^\infty([a, b]; \R^n)$ dé un valor extremo del funcional $S$, donde
	\begin{equation}\label{eq: action}
		S(\vec c) \coloneqq \int_{a}^b \mathcal L(\lambda, \vec c, \dot{\vec c}) \, d\lambda,
	\end{equation}
	es que las funciones $c^\mu, \dot c^\mu, \mu = \overline{1, n}$, satisfagan la ecuación de Euler-Lagrange
	\begin{equation}
		\frac{d}{d\lambda}\frac{\partial \mathcal L}{\partial \dot c^\mu} - \frac{\partial \mathcal L}{\partial c^\mu} = 0.
	\end{equation}
	A la función $\mathcal L$ se le llama \emph{lagrangiano} \parencite{gelfand-1975}.
\end{theorem}

La primera elección sería elegir en \eqref{eq: action} al lagrangiano como
\begin{equation}
	\mathcal L(\lambda, \bgamma, \dot{\bgamma}) = \norm{\dot{\bgamma}} = \sqrt{g_{\mu\nu} \dot c^\mu \dot c^\nu},
\end{equation}
donde $\bgamma(a) = \vec p$ y $\bgamma(b) = \vec q$.
Sin embargo, la raíz cuadrada anterior introduce complicaciones innecesarias en el cálculo de las ecuaciones de Euler-Lagrange. Resulta entonces más conveniente definir el funcional de energía $E: {\mathcal T}_{\vec p, \vec q}(\surface) \to \R$ como \parencite{carmo-1974}
\begin{equation}
	E(\bgamma) \coloneqq \frac{1}{2}\int_a^b \norm{\dot{\bgamma}}^2 \, d\lambda.
\end{equation}
Mostramos que las curvas $\vec c$ que extremizan a $\mathscr L$ también extremizan a $E$.
De la desigualdad de Cauchy-Bunyakovsky-Schwarz para integrales
\begin{equation}
	\bigl( \mathscr L(\bgamma)\bigr)^2 \le \int_{a}^{b} \bigl(\sqrt{
			g_{\mu\nu} \dot c^\mu \dot c^\nu
		}\bigr)^2\, d\lambda \int_a^b d\lambda
		= 2(b-a)E(\bgamma),
\end{equation}
donde la igualdad se cumple si $\norm{\dot{\bgamma}}$ es constante. Si el intervalo de intgración es $[0,1]$, entonces $\bgamma$ está normalizada y $\bigl( \mathscr L(\bgamma)\bigr)^2 = 2E(\bgamma)$.

\begin{lemma}
	Sean $\vec p,\vec q \in \surface$ y $ \hat\bgamma \in \mathcal T_{\vec p, \vec q}(\surface)$ una curva regular geodésica de rapidez constante. Entonces para todas las curvas $\bgamma \in \mathcal T_{\vec p, \vec q}(\surface)$ se cumple
	\begin{equation}
		E(\hat\bgamma) \le E(\bgamma).
	\end{equation}
\end{lemma}
\begin{proof}
	De la desigualdad entre $\mathscr L(\bgamma)$ y $E(\bgamma)$,
	\begin{equation}
		2(b-a)E(\hat\bgamma) = \bigl( \mathscr L(\hat\bgamma)\bigr)^2 \le \bigl( \mathscr L(\bgamma)\bigr)^2 \le 2(b-a)E(\bgamma).
	\end{equation}
\end{proof}

Tomemos entonces una curva regular $\bgamma$ en $\surface$ como en la ecuación \eqref{eq: gcurve}, donde $I = [0,1]$.
\begin{equation}
	E(\bgamma) \coloneqq \frac{1}{2}\int_0^1 \norm{\dot{\bgamma}}^2 \, d\lambda.
\end{equation}

\subsection{Ecuaciones de Euler-Lagrange y de la Geodésica}
El lagrangiano que estaremos empleando es, de forma totalmente explícita
\begin{equation}\label{eq: lagrangian}
	\mathcal L = \frac12g_{\mu\nu}(\vec c(\lambda)) \dot c^\mu(\lambda) \dot c^\nu(\lambda).
\end{equation}
Calculamos los términos de las ecuaciones de Euler-Lagrange
\begin{equation}
	\partialD{\mathcal L}{\dot c^\alpha} = \frac12 g_{\mu\nu} \partialD{}{\dot c^\alpha}(\dot c^\mu \dot c^\nu)
	= \frac12 g_{\mu\nu}(\dot c^\mu \delta\indices{^\nu_\alpha}  +\delta\indices{^\mu_\alpha} \dot c^\nu )
	= \frac12 g_{\mu\alpha}\dot c^\mu + \frac12 g_{\alpha\nu}\dot c^\nu.
\end{equation}
Por la simetría de las componentes $g_{\mu\nu}$ y haciendo el cambio $\nu \to \mu$ en el segundo término,
\begin{equation}
	\partialD{\mathcal L}{\dot c^\alpha} = \frac12 g_{\alpha\mu}\dot c^\mu + \frac12 g_{\alpha\mu}\dot c^\mu
	= g_{\alpha\mu}\dot c^\mu.
\end{equation}
Ahora ilustramos la necesidad de definir la derivada covariante.
\begin{equation}
\begin{split}
	\frac{d}{d\lambda}\partialD{\mathcal L}{\dot c^\alpha}
	&= g_{\alpha\mu}\ddot c^\mu + \dot g_{\alpha\mu}\dot c^\mu
	= g_{\alpha\mu}\ddot c^\mu + \dot c^\mu\dot c^\nu g_{\alpha\mu,\nu}\\
	 &= g_{\alpha\mu}\ddot c^\mu + \dot c^\mu\dot c^\nu(\innerp{\patch_{,\alpha}}{\patch_{,\mu,\nu}} + \innerp{\patch_{,\alpha, \nu}}{\patch_{,\mu}})\\
	&=g_{\alpha\mu}\ddot c^\mu +  \dot c^\mu\dot c^\nu  \left(
	\innerp*{\patch_{,\alpha}}{\patch_{,\mu; \nu} + \innerp{\patch_{,\mu,\nu}}{\vec N}\vec N} + \innerp*{\patch_{,\alpha; \nu} + \innerp{\patch_{,\alpha,\nu}}{\vec N}\vec N}{\patch_{,\mu}}
	\right)\\
	&= g_{\alpha\mu}\ddot c^\mu +  \dot c^\mu\dot c^\nu  \left(
	\innerp*{\patch_{,\alpha}}{\patch_{,\mu; \nu} } + \innerp*{\patch_{,\alpha; \nu}}{\patch_{,\mu}}
	\right)\\
	&=  g_{\alpha\mu}\ddot c^\mu +  \dot c^\mu\dot c^\nu 
	\innerp*{\patch_{,\alpha}}{\patch_{,\mu; \nu} } + \dot c^\mu\dot c^\nu \innerp*{\patch_{,\alpha; \nu}}{\patch_{,\mu}}.
\end{split}
\end{equation}
Aprovechando la simetría del producto interno y de la derivada covariante.
\begin{equation}
	\frac{d}{d\lambda}\partialD{\mathcal L}{\dot c^\alpha}
	= g_{\alpha\mu}\ddot c^\mu +  \dot c^\mu\dot c^\nu 
	\innerp*{\patch_{,\alpha}}{\patch_{,\mu; \nu} } + \dot c^\mu\dot c^\nu \innerp*{\patch_{,\mu}}{\patch_{,\nu; \alpha} }.
\end{equation}
El cálculo del siguiente término es
\begin{equation}
\begin{split}
	\partialD{\mathcal L}{c^\alpha} &= \frac12 \dot c^\mu \dot c^\nu \partialD{g_{\mu\nu}}{c^\alpha}
	=  \frac12 \dot c^\mu \dot c^\nu  (\innerp{\patch_{,\mu}}{\patch_{,\nu,\alpha}} + \innerp{\patch_{,\mu, \alpha}}{\patch_{,\nu}})\\
	&= \frac12 \dot c^\mu \dot c^\nu \left(
	\innerp*{\patch_{,\mu}}{\patch_{,\nu; \alpha}} + \innerp*{\patch_{,\mu; \alpha}}{\patch_{,\nu}}
	\right)\\
	&=  \frac12 \dot c^\mu \dot c^\nu \innerp*{\patch_{,\mu}}{\patch_{,\nu; \alpha}}
	+   \frac12 \dot c^\mu \dot c^\nu\innerp*{\patch_{,\mu; \alpha}}{\patch_{,\nu}}.
\end{split}
\end{equation}
Intercambiamos los índices mudos $\mu$ y $\nu$ en el segundo término.
\begin{equation}
	\partialD{\mathcal L}{c^\alpha} = \frac12 \dot c^\mu \dot c^\nu \innerp*{\patch_{,\mu}}{\patch_{,\nu; \alpha}}
	+   \frac12 \dot c^\mu \dot c^\nu \innerp*{\patch_{,\mu}}{\patch_{,\nu; \alpha}}
	= \dot c^\mu \dot c^\nu \innerp*{\patch_{,\mu}}{\patch_{,\nu; \alpha}}.
\end{equation}

De la ecuación de Euler-Lagrange obtenemos
\begin{equation}
	g_{\alpha\mu}\ddot c^\mu +  \dot c^\mu\dot c^\nu 
	\innerp*{\patch_{,\alpha}}{\patch_{,\mu; \nu} } + \dot c^\mu\dot c^\nu \innerp*{\patch_{,\mu}}{\patch_{,\nu; \alpha} } - \dot c^\mu \dot c^\nu \innerp*{\patch_{,\mu}}{\patch_{,\nu; \alpha}}
	= g_{\alpha\mu}\ddot c^\mu +  \dot c^\mu\dot c^\nu 
	\innerp*{\patch_{,\alpha}}{\patch_{,\mu; \nu} } = 0.
\end{equation}
Sustituyendo la ecuación \eqref{eq: covariant}
\begin{equation}
	 g_{\alpha\mu}\ddot c^\mu +  \dot c^\mu\dot c^\nu 
	\innerp*{\patch_{,\alpha}}{\patch_{,\mu; \nu} }
	=  g_{\alpha\mu}\ddot c^\mu +  \ChrisSym{\beta}{\mu\nu} \dot c^\mu\dot c^\nu 
	\innerp*{\patch_{,\alpha}}{\patch_{,\beta} },
\end{equation}
intercambiamos los índices $\beta \leftrightarrow \mu$ en el segundo término.
\begin{equation}
	 g_{\alpha\mu}\ddot c^\mu +  \ChrisSym{\beta}{\mu\nu} \dot c^\mu\dot c^\nu 
	\innerp*{\patch_{,\alpha}}{\patch_{,\beta} }
	= g_{\alpha\mu}\ddot c^\mu +  \ChrisSym{\mu}{\beta\nu} \dot c^\beta\dot c^\nu 
	\innerp*{\patch_{,\alpha}}{\patch_{,\mu} }
	= g_{\alpha\mu}\left( \ddot c^\mu +  \ChrisSym{\mu}{\beta\nu} \dot c^\beta\dot c^\nu  \right) = 0.
\end{equation}
Por la propiedad del difeomorfismo $\patch$ de que para toda $ \vec p \in \surface$ el plano $T_{\vec p}\surface$ es bidimensional y $\{\patch_{,1}(\vec q), \patch_{,2}(\vec p)\}$ es linealmente independiente, si todas las componentes de la métrica se anularan en algún punto $\vec p \in \surface$, entonces $g_{\mu\mu}(\vec p) = \norm{\patch_{,\mu}(\vec p)}^2 = 0$ y $\patch_{,\mu}(\vec p) = \vec 0$, por lo que $\{\patch_{,1}(\vec q), \patch_{,2}(\vec p)\}$ ya no sería linealmente independiente, lo cual es una contradicción.

Podemos entonces enunciar lo siguiente.
\begin{theorem}
	Sean $\vec p, \vec q$ puntos en una superficie regular $\mathcal S$ y $\patch : U \to \surface$ un parche coordenado cuya imagen contiene a los puntos $\vec p, \vec q$, donde $U \subset \R^2$ es abierto y conexo. Entonces en $\overline U$ existe una curva regular $\vec c$ tal que $\bgamma = \patch \circ \vec c$ es una geodésica normalizada que une a $\vec p$ y $\vec q$.
	
	La curva $\vec c$ satisface la ecuación de la geodésica
	\begin{equation}\label{eq: geodesicEq}
		\ddot c^\alpha +  \ChrisSym{\alpha}{\mu\nu} \dot c^\mu\dot c^\nu = 0.
	\end{equation}
\end{theorem}

\begin{remark}
	Si escribimos la parametrización de $\vec c$ como $\vec c = (q^1(\lambda), q^2(\lambda))$, donde $q^1, q^2$ son las coordenadas locales de $\patch(U)$, podemos obtener la forma convencional de la ecuación de la geodésica.
	\begin{equation}
		\ddot q^\alpha +  \ChrisSym{\alpha}{\mu\nu} \dot q^\mu\dot q^\nu = 0.
	\end{equation}
\end{remark}

\begin{remark}
	Lo que vuelve a la ecuación \eqref{eq: geodesicEq} extraordinaria es que la curva que debe hallarse para formar una geodésica se encuentra dada en las coordenadas locales y no en términos de las coordenadas $x, y, z$ de $\R^3$ usadas en $\surface$.
\end{remark}

\section{Aplicaciones}
En la práctica puede resultar engorroso calcular \textit{a priori} los símbolos de Christoffel, mientras que resulta más sencillo calcular las componentes de la métrica $g_{\mu\nu}$ dado un parche coordenado $\patch$. Por lo tanto, es más eficiente aplicar las ecuaciones de Euler-Lagrange directamente a \eqref{eq: lagrangian}.

\paragraph{Prueba de Cordura}
Antes de aplicar la ecuación de la geodésica a superficies regulares generales, verificaremos que la ecuación \eqref{eq: geodesicEq} coincide con un resultado bien establecido: las geodésicas en el plano.

El parche coordenado evidente es $\patch(x, y) = (x, y, z)$, donde $z \in \R$ es constante. Entonces
\begin{equation}
	\patch_{,x} = \univ 1, \qquad \patch_{,y} = \univ 3.
\end{equation}
Es inmediato que $\ChrisSym{\alpha}{\mu\nu} = 0$ para todos las combinaciones de $\alpha, \mu,\nu = 1, 2$.
Las ecuaciones geodésicas son entonces
\begin{equation}
	\ddot x = \ddot y = 0.
\end{equation}
Estas son las ecuaciones de una línea recta.

\subsection{Geodésicas en la Esfera}
Para el parche coordenado de una esfera de radio $R > 0$ empleamos las coordenas esféricas.
\begin{equation}
	\patch(\theta, \phi) = R(\sen \phi \cos \theta, \sen \phi \sen \theta, \cos \phi).
\end{equation}
Calculamos las componentes de la métrica.
\begin{equation}
	\patch_{,\theta} = R(-\sen \phi \sen \theta, \sen \phi \cos \theta,0), \qquad
	\patch_{,\phi} = R(\cos \phi \cos \theta, \cos \phi \sen \theta, -\sen\phi)
\end{equation}
Luego
\begin{gather}
	g_{\theta\theta} = R^2(\sen^2\phi \sen^2\theta + \sen^2\phi \cos^2 \theta)
	= R^2\sen^2\phi, \\
		g_{\theta\phi} = R^2(-\sen\phi \cos \phi \sen \theta \cos \theta + \sen\phi \cos \phi \sen \theta \cos \theta) = 0 = g_{\phi\theta},\\
	g_{\phi\phi} = R^2(\cos^2\phi \cos^2\theta + \cos^2\phi \sen^2 \theta + \sen^2\phi) = R^2.
\end{gather}
Omitiendo el parámetro, el lagrangiano relevante es
\begin{equation}
	\mathcal L = \frac12 (g_{\phi\phi}\dot \phi^2 + g_{\theta\theta} \dot\theta^2)
	= \frac{R^2}{2}(\dot \phi^2 +  \dot \theta^2\sen^2\phi).
\end{equation}
Al aplicar las ecuaciones de Euler-Lagrange para $\theta$ obtenemos
\begin{equation}
	\partialD{\mathcal L}{\theta} = 0, \qquad
	 \partialD{\mathcal L}{\dot \theta} = R^2 \dot \theta \sen^2\phi, \qquad
	 \frac{d}{d\lambda}\partialD{\mathcal L}{\dot \theta}  = R^2 (\ddot \theta\sen^2\phi  + \sen(2\phi)\dot \phi \dot \theta).
\end{equation}
Luego, para la coordenada $\theta$ la ecuación geodésica es
\begin{equation}
	R^2 ( \ddot \theta\sen^2\phi  + \sen(2\phi)\dot \phi \dot \theta)  = 0 \Longrightarrow \ddot \theta + 2 \dot \theta \dot \phi \cot \phi= 0.
\end{equation}

Para $\phi$, las ecuaciones son
\begin{equation}
	\partialD{\mathcal L}{\phi} = \frac{R^2}{2}\sen(2\phi)\dot \theta^2, \qquad
	 \partialD{\mathcal L}{\dot \phi} = R^2 \dot \phi, \qquad
	 \frac{d}{d\lambda}\partialD{\mathcal L}{\dot \phi}  = R^2 \ddot \phi.
\end{equation}
Luego,
\begin{equation}
	R^2 \ddot \phi - \frac{R^2}{2}\sen(2\phi)\dot \theta^2 = 0 \Longrightarrow \ddot \phi - \frac{1}{2}\sen(2\phi)\dot \theta^2 = 0.
\end{equation}

\begin{remark}
De estas ecuaciones podemos obtener de forma inmediata los símbolos de Christoffel para coordenadas esféricas.
\begin{equation}
	\ChrisSym{\theta}{\theta\theta} = \ChrisSym{\theta}{\phi\phi} = 0, \qquad \ChrisSym{\theta}{\theta\phi} = \cot \phi, \qquad \ChrisSym{\phi}{\theta\theta} = -\frac12 \sen (2\phi), \qquad \ChrisSym{\phi}{\phi\phi} = \ChrisSym{\phi}{\theta\phi} = 0.
\end{equation}
\end{remark}

Emplearemos el siguiente resultado de análsis.
\begin{proposition}
	Sea $\vec c \in C^1(I, \R^2)$ una curva regular que parametriza a la gráfica de la función $y = f(x)$. Si $\vec c(\lambda) = (x(\lambda), y(\lambda))$, entonces
	\begin{equation}
		y'(x(\lambda)) = \frac{dy}{dx}(x(\lambda)) = \frac{\dot y (\lambda)}{\dot x(\lambda)}.
	\end{equation}
\end{proposition}

Podemos escribir las ecuaciones geodésicas en términos de la métrica como
\begin{equation}
	\ddot \theta + \frac{g_{\theta\theta,\phi}}{g_{\theta\theta}}\dot\theta\dot\phi = 0, \qquad \ddot \phi - \frac{g_{\theta\theta,\phi}}{2R^2}\dot \theta^2 = 0.
\end{equation}
Podemos combinarlas como
\begin{equation}
	\begin{split}
		\ddot \theta \dot \phi - \dot\theta\ddot \phi + \frac{g_{\theta\theta,\phi}}{g_{\theta\theta}}\dot\theta\dot\phi^2 + \frac{g_{\theta\theta,\phi}}{2R^2}\dot \theta^3 &= 0\\
		\frac{\dot \phi}{\dot \theta}\frac{\ddot \theta \dot \phi - \dot\theta\ddot \phi}{\dot \theta^2} + \left(\frac{g_{\theta\theta,\phi}}{g_{\theta\theta}}\dot\phi \right)\frac{\dot \phi^2}{\dot \theta^2} + \frac{g_{\theta\theta,\phi}}{2R^2}\dot \phi &= 0\\
		2g^2_{\theta\theta}\frac{\dot \phi}{\dot \theta} \frac{d}{d\lambda}\frac{\dot \phi}{\dot \theta} + 2g_{\theta\theta} g_{\theta\theta,\phi}\dot \phi \left(\frac{\dot \phi}{\dot \theta}\right)^2 + \frac{g^2_{\theta\theta}g_{\theta\theta,\phi}}{R^2}\dot \phi &= 0\\
		g^2_{\theta\theta} \frac{d}{d\lambda}\left(\frac{\dot \phi}{\dot \theta}\right)^2
		+ \frac{d}{d\lambda}g^2_{\theta\theta}\left(\frac{\dot \phi}{\dot \theta}\right)^2
		+ \frac{d}{d\lambda}\frac{g^3_{\theta\theta}}{3R^2} &= 0.
	\end{split}
\end{equation}
Si denotamos a $\phi'(\theta) = d\phi/d\theta$, por la regla del producto y por la proposición anterior obtenemos la primera integral de la geodésica.
\begin{equation}
	\frac{d}{d\lambda}\left(g^2_{\theta\theta}\phi'^2 + \frac{g^3_{\theta\theta}}{3R^2} \right) = 0.
\end{equation}
Una solución inmediata es $\phi = \pi/2$, pues $\phi' = 0$ y $g_{\theta\theta} = R^2\sen^2(\pi/2) = 1$. Esto implicaría que la curva sería un segmento del ecuador de la esfera: la geodésica es un arco de un \emph{círculo máximo}.


\newpage

\printbibliography
\nocite{}



\end{document}
