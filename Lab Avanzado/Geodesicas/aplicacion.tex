\section{Aplicaciones}
En la práctica puede resultar engorroso calcular \textit{a priori} los símbolos de Christoffel, mientras que resulta más sencillo calcular las componentes de la métrica $g_{\mu\nu}$ dado un parche coordenado $\patch$. Por lo tanto, es más eficiente aplicar las ecuaciones de Euler-Lagrange directamente a \eqref{eq: lagrangian}.

\paragraph{Prueba de Cordura}
Antes de aplicar la ecuación de la geodésica a superficies regulares generales, verificaremos que la ecuación \eqref{eq: geodesicEq} coincide con un resultado bien establecido: las geodésicas en el plano.

El parche coordenado evidente es $\patch(x, y) = (x, y, z)$, donde $z \in \R$ es constante. Entonces
\begin{equation}
	\patch_{,x} = \univ 1, \qquad \patch_{,y} = \univ 3.
\end{equation}
Es inmediato que $\ChrisSym{\alpha}{\mu\nu} = 0$ para todos las combinaciones de $\alpha, \mu,\nu = 1, 2$.
Las ecuaciones geodésicas son entonces
\begin{equation}
	\ddot x = \ddot y = 0.
\end{equation}
Estas son las ecuaciones de una línea recta.

\subsection{Geodésicas en la Esfera}
Para el parche coordenado de una esfera de radio $R > 0$ empleamos las coordenas esféricas.
\begin{equation}
	\patch(\theta, \phi) = R(\sen \phi \cos \theta, \sen \phi \sen \theta, \cos \phi).
\end{equation}
Calculamos las componentes de la métrica.
\begin{equation}
	\patch_{,\theta} = R(-\sen \phi \sen \theta, \sen \phi \cos \theta,0), \qquad
	\patch_{,\phi} = R(\cos \phi \cos \theta, \cos \phi \sen \theta, -\sen\phi)
\end{equation}
Luego
\begin{gather}
	g_{\theta\theta} = R^2(\sen^2\phi \sen^2\theta + \sen^2\phi \cos^2 \theta)
	= R^2\sen^2\phi, \\
		g_{\theta\phi} = R^2(-\sen\phi \cos \phi \sen \theta \cos \theta + \sen\phi \cos \phi \sen \theta \cos \theta) = 0 = g_{\phi\theta},\\
	g_{\phi\phi} = R^2(\cos^2\phi \cos^2\theta + \cos^2\phi \sen^2 \theta + \sen^2\phi) = R^2.
\end{gather}
Omitiendo el parámetro, el lagrangiano relevante es
\begin{equation}
	\mathcal L = \frac12 (g_{\phi\phi}\dot \phi^2 + g_{\theta\theta} \dot\theta^2)
	= \frac{R^2}{2}(\dot \phi^2 +  \dot \theta^2\sen^2\phi).
\end{equation}
Al aplicar las ecuaciones de Euler-Lagrange para $\theta$ obtenemos
\begin{equation}
	\partialD{\mathcal L}{\theta} = 0, \qquad
	 \partialD{\mathcal L}{\dot \theta} = R^2 \dot \theta \sen^2\phi, \qquad
	 \frac{d}{d\lambda}\partialD{\mathcal L}{\dot \theta}  = R^2 (\ddot \theta\sen^2\phi  + \sen(2\phi)\dot \phi \dot \theta).
\end{equation}
Luego, para la coordenada $\theta$ la ecuación geodésica es
\begin{equation}
	R^2 ( \ddot \theta\sen^2\phi  + \sen(2\phi)\dot \phi \dot \theta)  = 0 \Longrightarrow \ddot \theta + 2 \dot \theta \dot \phi \cot \phi= 0.
\end{equation}

Para $\phi$, las ecuaciones son
\begin{equation}
	\partialD{\mathcal L}{\phi} = \frac{R^2}{2}\sen(2\phi)\dot \theta^2, \qquad
	 \partialD{\mathcal L}{\dot \phi} = R^2 \dot \phi, \qquad
	 \frac{d}{d\lambda}\partialD{\mathcal L}{\dot \phi}  = R^2 \ddot \phi.
\end{equation}
Luego,
\begin{equation}
	R^2 \ddot \phi - \frac{R^2}{2}\sen(2\phi)\dot \theta^2 = 0 \Longrightarrow \ddot \phi - \frac{1}{2}\sen(2\phi)\dot \theta^2 = 0.
\end{equation}

\begin{remark}
De estas ecuaciones podemos obtener de forma inmediata los símbolos de Christoffel para coordenadas esféricas.
\begin{equation}
	\ChrisSym{\theta}{\theta\theta} = \ChrisSym{\theta}{\phi\phi} = 0, \qquad \ChrisSym{\theta}{\theta\phi} = \cot \phi, \qquad \ChrisSym{\phi}{\theta\theta} = -\frac12 \sen (2\phi), \qquad \ChrisSym{\phi}{\phi\phi} = \ChrisSym{\phi}{\theta\phi} = 0.
\end{equation}
\end{remark}

Emplearemos el siguiente resultado de análsis.
\begin{proposition}
	Sea $\vec c \in C^1(I, \R^2)$ una curva regular que parametriza a la gráfica de la función $y = f(x)$. Si $\vec c(\lambda) = (x(\lambda), y(\lambda))$, entonces
	\begin{equation}
		y'(x(\lambda)) = \frac{dy}{dx}(x(\lambda)) = \frac{\dot y (\lambda)}{\dot x(\lambda)}.
	\end{equation}
\end{proposition}

Podemos escribir las ecuaciones geodésicas en términos de la métrica como
\begin{equation}
	\ddot \theta + \frac{g_{\theta\theta,\phi}}{g_{\theta\theta}}\dot\theta\dot\phi = 0, \qquad \ddot \phi - \frac{g_{\theta\theta,\phi}}{2R^2}\dot \theta^2 = 0.
\end{equation}
Podemos combinarlas como
\begin{equation}
	\begin{split}
		\ddot \theta \dot \phi - \dot\theta\ddot \phi + \frac{g_{\theta\theta,\phi}}{g_{\theta\theta}}\dot\theta\dot\phi^2 + \frac{g_{\theta\theta,\phi}}{2R^2}\dot \theta^3 &= 0\\
		\frac{\dot \phi}{\dot \theta}\frac{\ddot \theta \dot \phi - \dot\theta\ddot \phi}{\dot \theta^2} + \left(\frac{g_{\theta\theta,\phi}}{g_{\theta\theta}}\dot\phi \right)\frac{\dot \phi^2}{\dot \theta^2} + \frac{g_{\theta\theta,\phi}}{2R^2}\dot \phi &= 0\\
		2g^2_{\theta\theta}\frac{\dot \phi}{\dot \theta} \frac{d}{d\lambda}\frac{\dot \phi}{\dot \theta} + 2g_{\theta\theta} g_{\theta\theta,\phi}\dot \phi \left(\frac{\dot \phi}{\dot \theta}\right)^2 + \frac{g^2_{\theta\theta}g_{\theta\theta,\phi}}{R^2}\dot \phi &= 0\\
		g^2_{\theta\theta} \frac{d}{d\lambda}\left(\frac{\dot \phi}{\dot \theta}\right)^2
		+ \frac{d}{d\lambda}g^2_{\theta\theta}\left(\frac{\dot \phi}{\dot \theta}\right)^2
		+ \frac{d}{d\lambda}\frac{g^3_{\theta\theta}}{3R^2} &= 0.
	\end{split}
\end{equation}
Si denotamos a $\phi'(\theta) = d\phi/d\theta$, por la regla del producto y por la proposición anterior obtenemos la primera integral de la geodésica.
\begin{equation}
	\frac{d}{d\lambda}\left(g^2_{\theta\theta}\phi'^2 + \frac{g^3_{\theta\theta}}{3R^2} \right) = 0.
\end{equation}
Una solución inmediata es $\phi = \pi/2$, pues $\phi' = 0$ y $g_{\theta\theta} = R^2\sen^2(\pi/2) = 1$. Esto implicaría que la curva sería un segmento del ecuador de la esfera: la geodésica es un arco de un \emph{círculo máximo}.