En la presente práctica de laboratorio se estudió el comportamiento de la luz al pasar a través de lentes delgadas. Las lentes son elementos ópticos fundamentales que permiten manipular la dirección y las propiedades de los rayos luminosos mediante fenómenos de refracción. A partir de la relación entre la forma de la lente y el índice de refracción del material, se generan efectos ópticos que resultan de gran utilidad en aplicaciones como la corrección visual, instrumentos de precisión y dispositivos ópticos.

El objetivo de esta práctica fue observar cómo los diferentes tipos de lentes delgadas, como las convergentes y divergentes, afectan la trayectoria de la luz, y cómo sus características geométricas influyen en la formación de imágenes. Se analizaron conceptos como la distancia focal, el aumento y la formación de imágenes virtuales y reales. También se exploraron algunos sistemas ópticos simples y compuestos que pueden ser construidos a partir de lentes delgadas.