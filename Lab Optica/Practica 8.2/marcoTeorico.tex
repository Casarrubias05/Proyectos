\section{Polarización: una descripción matemática}\label{sec: teo}

Para estudiar la propagación de la luz como una onda electromagnética consideramos el caso especial de las ecuaciones de Maxwell en el vacío y sin fuentes.
\begin{align}
    \div E &= 0, & \rot E + \partial_t \vec B &= 0\\
    \div B &= 0, & \rot B - \frac{1}{c^2}\partial_t \vec E &= 0.
\end{align}

Al desacoplarlas obtenemos que los campos $\vec E$ y $\vec B$ satisfacen ecuaciones de onda
\begin{equation}
    \nabla^2 \vec E = \frac{1}{c^2}\partial^2_t \vec E, \qquad \nabla^2 \vec B = \frac{1}{c^2}\partial^2_t \vec B.
\end{equation}

Lejos de las fuentes, podemos aproximar a estas ondas como ondas planas con frecuencia angular $\omega$ que se propagan en la dirección $\vec k$.
\begin{gather}
    \vec E(\r, t) = \vec E_0 \cos(\dotp k r - \omega t),\\
    \vec B(\r, t) = \vec B_0 \cos(\dotp k r - \omega t),
\end{gather}
donde $\norm E = c\norm B$.

Para estudiar la polarización de la luz, como esta está definida en términos del campo eléctrico, reescribimos la ecuación de onda para el campo eléctrico en notación compleja
\begin{equation}
    \tilde{\vec E}(\r, t) = \tilde{\vec E}_0 \eto{i(\dotp k r - \omega t)},
\end{equation}
donde $\tilde{\vec E}_0$ es un vector complejo. En esta ecuación el campo eléctrico físico se obtiene tomando la parte real de $\tilde{\vec E}$.

Si tomamos al eje $z$ como la dirección de propagación de la onda, entonces podemos escribir
\begin{equation}\label{eq: complexE}
    \tilde{\vec E} = \eto{i(kz - \omega t)}(\tilde{E}_{0x}  \vec i + \tilde{E}_{0y}  \vec j),
\end{equation}
y al expresar a las componentes en forma polar
\begin{equation}\label{eq: amplitudes}
    \tilde{E}_{0x} = E_{0x}\eto{i\phi_x},
    \tilde{E}_{0y} = E_{0y}\eto{i\phi_y},
\end{equation}
podemos obtener los distintos tipos de polarización. Si $\phi_x = \phi_y = \phi$, entonces
\begin{gather}
    \tilde{\vec E} = \eto{i(kz - \omega t + \phi)}(E_{0x}  \vec i + E_{0y}  \vec j),\\
    \vec E = \cos(kz - \omega t + \phi)(E_{0x}  \vec i + E_{0y}  \vec j),
\end{gather}
es decir, $\vec E$ oscila en una sola dirección y decimos entonces que está \emph{linealmente polarizada}.

Si $\phi_y = \phi_x - \pi/2$,
\begin{equation}\label{eq: elliptical}
    \tilde{\vec E} = \eto{i(kz - \omega t + \phi_x)}(E_{0x}  \vec i - i E_{0y}  \vec j),
\end{equation}
\begin{multline}
    \vec E = E_{0x}\cos(kz - \omega t + \phi_x)\vec i\\
    + E_{0y}\sen(kz - \omega t + \phi_x)\vec j,
\end{multline}
y el campo eléctrico rota en sentido antihorario sobre una elipse y decimos que tiene una \emph{polarización elíptica izquierda}; la polarización elíptica derecha se obtiene cuando $\phi_y = \phi_x + \pi/2$, y si a la vez $E_{0x} = E_{0y} = E_0$,
\begin{equation}
    \vec E = E_0(\cos(kz - \omega t + \phi_x) \vec i + \sen(kz - \omega t + \phi_x)  \vec j),
\end{equation}
y en este caso el campo eléctrico rota sobre una circunferencia y decimos que tiene una polarización \emph{circular}.

Experimentalmente no se mide como varía el campo eléctrico, pues para el espectro visible las frecuencias $\omega$ son del orden de \qty{1e15}{\Hz}. Entonces, al hacer una medición lo que se está midiendo es el promedio de la potencia por unidad de área que transmite la onda, o \emph{irradiancia} $I$ que se calcula como
\begin{equation}
    I = \braket{S},
\end{equation}
donde $S$ es la magnitud del vector de Poynting
\begin{equation}
    \vec S \coloneq \frac{1}{\mu_0} \crossp{E}{B}.
\end{equation}
Para ondas monocromáticas, como las del láser rojo que empleamos,
\begin{equation}
    \vec S = \frac{1}{c\mu_0} E^2 \vec k = c\epsilon_0 E_0^2 \cos^2(kz - \omega t + \phi) \vec k.
\end{equation}
Entonces
\begin{equation}\label{eq: irradiance}
    I = \frac12 c\epsilon_0 E_0^2.
\end{equation}

\subsection{Ley de Malus}

En la primera parte de la práctica introducimos un polarizador lineal entre la fuente, el láser rojo, y el medidor de potencia. Un polarizador lineal es un elemento óptico que, a través de diversos mecanismos como sería el \emph{dicroísmo}, la reflexión, la dispersión o la doble reflexión, separa las polarizaciones de la onda eléctrica y transmite solamente una de ellas \parencite{hecht-1999}. Esta separación se observa en que existe una dirección preferencial del polarizador que permite la máxima transmisión de luz polarizada lineal incidente en él; esta dirección se denomina \emph{eje de transmisión} \parencite{fowles-1989}. Un polarizador ideal es uno que transmite totalmente a luz polarizada paralelamente a su eje de transmisión y que absorbe totalmente luz polarizada perpendicularmente a su eje.

Este eje de transmisión define el subespacio unidimensional $U = \Span \univ n$, donde $\univ n$ es un vector unitario paralelo al eje de transmisión del polarizador lineal. Podemos entonces separar al espacio $\R^3$ como la suma directa $\R^3 = U\oplus U^{\perp}$, es decir, existen vectores únicos y ortogonales entre sí $\vec E_1 \in U$ y $\vec E_2 \in U^{\perp}$ tales que
\begin{equation}
    \vec E_0 = \vec E_1 + \vec E_2.
\end{equation}
El vector $\vec E_1$ se obtiene como
\begin{equation}\label{eq: proj}
    \vec E_1 = (\dotp{\vec E}{\univ n})\univ n = E\cos\phi \univ n,
\end{equation}
donde $\phi$ es el ángulo que forma $\vec E_0$ con $\univ n$. Este es el campo eléctrico que será detectado después de que haya atravesado al polarizador lineal. Sustituyendo en la ecuación \eqref{eq: irradiance}
\begin{equation}
    I(\phi) = \frac12 c\epsilon_0 (E_0\cos\phi)^2 = \frac12 c\epsilon_0 E_0^2 \cos^2\phi.
\end{equation}
Si definimos $I_0 = c\epsilon_0 E_0^2/2$, entonces obtenemos la \emph{Ley de Malus}
\begin{equation}\label{eq: Malus}
    I(\theta) = I_0 \cos^2\phi.
\end{equation}
Con esta ecuación podemos determinar, hasta un múltiplo entero de $\pi$, el eje de transmisión de un polarizador lineal, siempre y cuando la luz incidente esté polarizada linealmente.

Si la luz incidente no está polarizada linealmente, o tiene una polarización circular, entonces todos los valores de $\phi$ ocurren con igual probabilidad y lo que se mide es el promedio sobre todos los ángulos. Como $\braket{\cos^2 \phi} = 1/2$, entonces la ley de Malus es
\begin{equation}
     I(\phi) = \frac{I_0}{2}.
\end{equation}

\subsection{Cálculo de Jones: Álgebra Lineal en la Óptica}

Tomando nuevamente una onda electromagnética que se propaga en la dirección $z$, y usando la descripción compleja, es apropiado entonces usar el espacio vectorial $\C^2$ con su base canónica y el producto interno Euclídeo para describir las ondas y su polarización en el plano.

Cuando obtuvimos las expresiones para las polarizaciones lineales, elípticas y circulares obtuvimos que, al pasar al campo real $\vec E$, el estado de polarización estaba determinado en $\tilde{\vec E}$ por las amplitudes y sus fases en la ecuaciones \eqref{eq: amplitudes}. Esto, aunado a que en la expresión \eqref{eq: complexE} podemos separar el vector de amplitudes del factor $\eto{i(kz-\omega t)}$, nos lleva a la representación
\begin{equation}\label{eq: MatrixE}
    \tilde{\vec E} = \Matrix{E_{0x}\eto{i\phi_x} \\ E_{0y}\eto{i\phi_y}}.
\end{equation}

En una onda polarizada linealmente es la condición sobre las fases es $\phi_x = \phi_y = \phi$ y su representación matricial es
\begin{equation}
    \eto{i\phi} \Matrix{E_{0x} \\ E_{0y}}.
\end{equation}
Estas componentes las podemos expresar como $E_{0x} = E_0 \cos \theta$ y $E_{0y} = E_0 \sen \theta$ y
\begin{equation}
     \eto{i\phi} \Matrix{E_{0x} \\ E_{0y}} =  E_0\eto{i\phi} \Matrix{ \cos\theta \\ \sen \theta}.
\end{equation}
Como solo el tipo de polarización es de interés y $E_0$ es arbitrario, podemos representar el estado de polarización lineal como
un vector normalizado
\begin{equation}
      \Matrix{ \cos\theta \\ \sen \theta} = \cos\theta\Matrix{1 \\ 0} + \sen\theta\Matrix{0 \\ 1}.
\end{equation}
De la expresión anterior podemos identificar los vectores de la base canónica con los estados de polarización lineal horizontal y vertical (usando la notación \emph{ket}) $\ket{H}$ y $\ket V$
\begin{equation}
    \ket H = \Matrix{1 \\ 0}, \qquad \ket V = \Matrix{0 \\ 1}.
\end{equation}
Por ejemplo, si la polarización es lineal con un ángulo de $\pi/4$, su representación es
\begin{equation}
    \ket D = \frac{1}{\sqrt 2}\ket H + \frac{1}{\sqrt 2}\ket V,
\end{equation}
donde la $D$ se refiere a que el vector $\vec E_0$ es diagonal.

La representación de la polarización elíptica izquierda se obtiene inmediatamente de \eqref{eq: elliptical}:
\begin{equation}
    \Matrix{E_{0x} \\ -iE_{0y}}.
\end{equation}
Por convención $E_{0x} = A$ y $E_{0y} = B$, y su forma normalizada es
\begin{equation}
    \frac{1}{\sqrt{A^2 + B^2}}\Matrix{A \\ -iB} = \frac{A\ket H - iB\ket V}{\sqrt{A^2 + B^2}}.
\end{equation}
En las polarizaciones elípticas que hemos descrito los ejes de la elipse están alineados con los ejes $x$ y $y$, lo cual ocurre porque la diferencia de fases es $\Delta \phi = \phi_y - \phi_x = \pm \pi/2$. Si permitimos que $\Delta \phi$ sea un número arbitrario, la representación de Jones es
\begin{equation}\label{eq: jonesE}
    \tilde{\vec E}_0 = \eto{i\phi_x} \Matrix{A \\ b\eto{i\Delta \phi}},
\end{equation}
por la identidad de Euler $b\eto{i\Delta \phi} = b\cos\Delta \phi + i b\sen \Delta \phi$ y tomando $B = b\cos \Delta \phi, C = b \sen \Delta \phi$
\begin{equation}
    \tilde{\vec E}_0 = \eto{i\phi_x} \Matrix{A \\ B + iC}.
\end{equation}
Normalizando
\begin{equation}
    \ket E = \frac{A\ket H + (B + iC)\ket V}{\sqrt{A^2 + B^2 + C^2}}.
\end{equation}
La polarización circular izquierda se obtiene tomando $\Delta \phi = -\pi/2$ y $A = b$ en \eqref{eq: jonesE}
\begin{equation}
    A\eto{i\phi_x} \Matrix{1 \\ \eto{-i\pi/2}} = A\eto{i\phi_x} \Matrix{1 \\ -i}.
\end{equation}
Normalizando
\begin{equation}
    \ket L = \frac{1}{\sqrt 2}\ket H - \frac{i}{\sqrt 2}\ket V.
\end{equation}

\subsection{Matrices de Jones}
Para pasar al cálculo matricial de Jones de la luz polarizada, daremos por hecho que los efectos de elementos ópticos sobre la luz, como son polarizadores lineales, rotadores y retardadores de fase por mencionar los relevantes a esta práctica, son lineales. Podemos entonces extender el uso de álgebra lineal de la deducción de la ley de Malus a describir las transformaciones que sufre el campo eléctrico al atravesar los elementos ópticos mencionados.

Un resultado de álgebra lineal es que, dada una base $\{\ket{\mu}_V\}_{\mu=1}^n$ de un espacio vectorial $V$ y un conjunto de vectores $\{\ket{\mu}_W\}_{\mu=1}^n$ en el espacio vectorial $W$, existe una transformación lineal única $T\in \mathcal L(V; W)$ tal que $T\ket{\mu}_V = \ket{\mu}_W$ para $\mu = \overline{1, n}$ \parencite{axler-2023}. Es decir, una transformación lineal está totalmente determinada por sus efectos sobre la base.

En la matriz $m\times n$ de la transformación $T$ respecto a las bases $\{\ket{\mu}_V\}_{\mu=1}^n$ de $V$ y $\{\ket{\mu}_W\}_{\mu=1}^m$ de $W$, las columnas son las matrices de coeficientes del vector $T \ket{\mu}_V$ respecto a la base $\{\ket{\mu}_W\}_{\mu=1}^m$. Hemos de recordar que las transformaciones para los elementos ópticos lineales son operadores del espacio $\C^2$, y que hemos tomado su base canónica.

En la discusión que nos llevó a la ley de Malus mencionamos que un polarizador lineal absorbe la componente perpendicular del campo eléctrico a su eje de transmisión y permite el paso de la componente paralela. No fue mencionado explícitamente, pero en la ecuación \eqref{eq: proj} la componente $\vec E_1$ es la proyección ortogonal de $\vec E$ sobre el subespacio $\Span \univ n$, y es obtenida como sigue: sea $U$ un subespacio de $V$ de dimensión finita; \emph{la proyección ortogonal de $V$ sobre $U$} es el operador $P_U$ definido como
\begin{equation}
    P_U(\ket \mu + {\ket \mu}^\perp) \coloneqq \ket \mu, \quad \ket \mu \in U, \quad {\ket \mu}^\perp \in U^\perp.
\end{equation}
Dada una base $\{\ket{\mu}\}_{\mu=1}^m$ ortonormal de $U$,
\begin{equation}
\begin{split}
    P_U \ket \lambda &= \sum_{\mu}\braket{\mu \vert \lambda}\ket \mu
    = \sum_{\mu} \ket \mu \braket{ \mu \vert \lambda}\\
    &= \left(\sum_{\mu} \ket \mu \bra \mu \right) \ket \lambda.
\end{split}
\end{equation}
Entonces, a un polarizador lineal le corresponde la transformación $P$. Si el eje de transmisión forma un ángulo $\theta$ con el eje $x$, paralelo al vector $\ket H$, su matriz es ($\univ n \mapsto \ket n = \cos \theta \ket H + \sen \theta \ket V$)
\begin{equation}\label{eq: simple_mat_mul}
\begin{split}
    &\Matrix{
        \braket{H \vert P \vert H} & \braket{H \vert P \vert V}\\
        \braket{V \vert P \vert H} & \braket{V \vert P \vert V}
    }\\
    &=
    \Matrix{
        \braket{H \vert n} \braket{n \vert H} & \braket{H \vert n} \braket{n \vert V}\\
        \braket{V \vert n} \braket{n \vert H} & \braket{V \vert n} \braket{n \vert V}
    }\\
    &=
    \Matrix{
        \cos^2 \theta & \sen \theta \cos \theta\\
        \sen \theta \cos \theta & \sen^2 \theta
    }.
\end{split}
\end{equation}
Si luz con polarización $-\sen \theta \ket H + \cos \theta \ket V$ (perpendicular al eje de transmisión) incide sobre el polarizador lineal, la polarización posterior será
\begin{equation}
\begin{split}
    &\Matrix{
        \cos^2 \theta & \sen \theta \cos \theta\\
        \sen \theta \cos \theta & \sen^2 \theta
    }\Matrix{-\sen \theta \\ \cos \theta}\\
    &= \Matrix{-\sen \theta  \cos^2 \theta + \sen \theta \cos^2\theta \\ -\sen^2\theta \cos\theta + \sen^2\theta \cos \theta} = 0.
\end{split}
\end{equation}
Es decir, la luz es extinguida por el polarizador, como se buscaba.

Un rotador, como su nombre lo indica, rota la dirección de polarización de luz linealmente polarizada incidente sobre él. La matriz que le corresponde es, evidentemente, la conocida matriz de rotación por un ángulo $\beta$
\begin{equation}
    \Matrix{\cos \beta & - \sen \beta \\ \sen \beta & \cos \beta}.
\end{equation}

Un retardador de fase altera la fase de cada componente en \eqref{eq: MatrixE}, sin alterar las magnitudes de las componentes. Es decir, buscamos una transformación $R$ que actúe como
\begin{equation}
\begin{split}
    R : \Matrix{E_{0x}\eto{i\phi_x} \\ E_{0y}\eto{i\phi_y}} &\mapsto \Matrix{E_{0x}\eto{i(\phi_x + \epsilon_x)} \\ E_{0y}\eto{i(\phi_y + \epsilon_y)}}\\
    &= \Matrix{\eto{i\epsilon_x}E_{0x}\eto{i\phi_x} \\ \eto{i\epsilon_y}E_{0y}\eto{i\phi_y}}.
\end{split}
\end{equation}
Entonces
\begin{equation}
    R : \Matrix{1 \\ 0} \mapsto \Matrix{\eto{i\epsilon_x} \\ 0}, \quad R : \Matrix{0 \\ 1} \mapsto \Matrix{0 \\ \eto{i\epsilon_y}}.
\end{equation}
Y su matriz es
\begin{equation}\label{eq: retarMat}
    \Matrix{
        \eto{i\epsilon_x} & 0 \\ 0 & \eto{i\epsilon_y}
    }.
\end{equation}

En los casos especiales en que la diferencia de fases inducida por el retardador sea $\Delta \phi = \phi_y - \phi_x$ sea $\pi/2$ o $\pi$, la placa que los produce se les denomina \emph{retardadores de cuarto de longitud de onda} y de \emph{media longitud de onda}, respectivamente.

Las diferencias de fase se producen porque los retardadores tienen dos índices de refracción: la componente del campo eléctrico paralela al eje óptico se propaga con velocidad $v_o = c/n_o$ y la componente perpendicular con $v_e = c/n_e$. La dirección en la cual el índice de refracción es menor se le denomina \emph{eje rápido}, y la otra dirección es el \emph{eje lento} \parencite{libretexts-2022}.
Esta diferencia en los índices de refracción produce que las componentes del campo eléctrico recorran distintos caminos ópticos, y esta diferencia en los caminos ópticos $\Delta$ se relaciona con la diferencia de fases a través de
\begin{equation}
    \Delta = \frac{\Delta \phi}{k} = \frac{\Delta \phi}{2\pi}\lambda.
\end{equation}
Entonces, si $\Delta \phi = \pi/2$, $\Delta = \lambda/4$, y si $\Delta \phi = \pi$, $\Delta = \lambda / 2$.