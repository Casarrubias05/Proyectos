\section{Resultados}
\subsection{Doble Rendija}
Para la doble rendija con $a = \qty{.2}{\mm}$ obtuvimos una distancia promedio entre los centros de las primera y segunda banda brillante de $\Delta x = \qty{.255}{\cm}$. Usando la ecuación \eqref{eq: maxima} con $n = 1$, donde $\sen \theta = \Delta x/r_0$, obtenemos un valor para $a$ de
\begin{equation}\label{eq: separation}
\begin{split}
	a &= \frac{r_0}{\Delta x} \lambda = \frac{\qty{114\pm.05}{cm}}{\qty{.255(1)}{\cm}}\times \qty{633}{\nm}\\
	&= \qty{.283(2)}{\mm}.
\end{split}
\end{equation}
Para $a = \qty{.3}{\mm}$, $\Delta x = \qty{.174}{\cm}$ y
\begin{equation}
\begin{split}
	a &= \frac{\qty{114\pm.05}{cm}}{\qty{.174(1)}{\cm}}\times \qty{633}{\nm}\\
	&= \qty{.415(3)}{\mm}.
\end{split}
\end{equation}
Para las mediciones $a =\qty{.45}{\mm}$.
\begin{equation}
\begin{split}
	a &= \frac{\qty{114\pm.05}{cm}}{\qty{.106(1)}{\cm}}\times \qty{633}{\nm}\\
	&= \qty{.681(7)}{\mm}.
\end{split}
\end{equation}

\subsection{Anillo de Airy}
Ahora comparamos el diámetro especificado de la apertura circular con el valor calculado con \eqref{eq: airy}.
Para $D = \qty{.2}{\mm}$, el diámetro medido del anillo de Airy fue $D_A = \qty{.57(1)}{\cm}$, por lo que $\sen \theta = D_A/2r_0$ y
\begin{equation}
	D = \frac{2.44 r_0}{D_A}\lambda.
\end{equation}
Entonces
\begin{equation}
\begin{split}
	D &= \frac{2.44 \times \qty{114\pm .05}{cm}}{\qty{.570(1)}{\cm}}\times \qty{633}{\nm}\\
	&= \qty{.309(1)}{\mm}.
\end{split}
\end{equation}
Con $D = \qty{.3}{\mm}$,
\begin{equation}
\begin{split}
	D &= \frac{2.44 \times \qty{114\pm .05}{cm}}{\qty{.520(1)}{\cm}}\times \qty{633}{\nm}\\
	&= \qty{.339(2)}{\mm}.
\end{split}
\end{equation}
Y cuando $D = \qty{.4}{\mm}$.
\begin{equation}
\begin{split}
	D &= \frac{2.44 \times \qty{114\pm .05}{cm}}{\qty{.436(1)}{\cm}}\times \qty{633}{\nm}\\
	&= \qty{.404(2)}{\mm}.
\end{split}
\end{equation}

\subsection{Grosor del Cabello}
El cabello descrito en la subsección \ref{sec: hair} crea un obstáculo equivalente a una doble rendija ``\emph{negativa}'', por lo que podemos emplear nuevamente la ecuación \eqref{eq: separation}. En promedio medimos una distancia entre los centros de las primeras dos bandas brillantes de $\Delta x = \qty{3.164(1)}{\cm}$. Entonces el grosor del cabello es
\begin{equation}
\begin{split}
	a &= \frac{\qty{278.3}{\cm}}{\qty{3.164(1)}{\cm}}\times \qty{633}{\nm}\\
	&= \qty{.0557(1)}{\mm} = \qty{55.7(1)}{\micro\m}.
\end{split}
\end{equation}
En la práctica 4 medimos cabellos con un microscopio con grosores de \qtylist{77.8; 82.2; 93.3}{\micro\m}, por lo que esta nueva medición parece ser un resultado aceptable.